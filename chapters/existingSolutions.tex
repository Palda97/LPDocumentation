If there already exists a perfect solution which suits our requirements, it doesn't make sense to create it one more time.
\todo[inline]{Jazyk! Nepoužívejte zkratky (doesn't). A opět, napište, co v sekci bude, ne coby kdyby. Tedy přehled existujících řešení. Nepište, že pokud existuje, nemá cenu ho dělat znovu. }

\todo[inline]{Takto bude sekce působit (a se stylystického pohledu bude hodnocena) jako nevyvážená. Je příliš krátká.}
\todo[inline]{Navíc používáte osamocené podnadpisy (tj. že zavádíte další úroveň nadpisu, např. 2.1.1, a přitom je v té úrovni jen jeden (není žádný 2.1.2). Tedy buďto úroveň nezavádějte, nebo dodejte další obsah pod další nadpis.}

\section{Description of existing solutions}
Here is a brief description of solutions that already exist.

\subsection{Responsive web app}
The web front-end can be used by any device possessing a web browser.
\todo[inline]{use formal language}
%It \todo[inline]{what is "it"?} works on any device \todo[inline]{so, a typewriter?}, not just android devices.
%\todo[inline]{this reads like a concept of a blog post, not like a formal message.}
Users don't have to download any application, which also means they don't have to update it.
The responsive web app only works with one server instance.
On an Android device, it responds slower to screen rotation and animations often lag.
% Even if you don't want to execute anything, just view history or list pipelines, you have to be online.
Even if users don't want to execute anything, just view history or list pipelines, they have to be online.
It is also browser dependent.

\section{Summary of existing solutions}
In this section, existing solutions are being compared with each other (see \autoref{tab:existingSolutions}), resulting in a decision if there is a need to create new application.
\todo[inline]{each heading needs to be followed by a text and not another heading. What is the contents of this section, briefly?}

\begin{table}[ht]\centering
\caption[Existing solutions]{Features of existing solutions}\label{tab:existingSolutions}
\begin{tabular}{l|l|l}
\hline
\textbf{Feature} & \textbf{Android App} & \textbf{Web App} \\ \hline
Can work with multiple server instances & + & - \\ \hline
Works on any device & - & + \\ \hline
Doesn't need to be downloaded & - & + \\ \hline
Smooth UI & + & - \\ \hline
Can view stuff while offline & + & - \\ \hline
\end{tabular}
\end{table}

The comparison indicates that the web application is missing at least one critical feature, that is not being able to work with multiple server instances, thus there is a need for the creation of new application.