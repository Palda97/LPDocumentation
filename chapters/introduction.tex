\ld{} \todo[inline]{link, např. \url{https://www.w3.org/standards/semanticweb/data}} is data that links to other data using URIs.
The URIs identify not only other data objects, but also relations, types or other concepts.
\etl{} is a system for working with \ld{}.
\todo[inline]{tady není jasné, co je to "working". dejte příklady - minimálně producing, transforming and consuming} \todo[inline]{tady to teď nenavazuje} by students and researchers from Prague computer science universities.
\todo[inline]{je taky třeba přidat příslušné reference. Tady konkrétně pomocí BibTeX dle \url{https://dblp.uni-trier.de/rec/conf/iiwas/KlimekS17.html?view=bibtex} - beztak máte dost nedostatečnou sekci bibliografie (jen 2 záznamy)}.
It is an open source project \cite{etl} with MIT license, so anyone can use it.
%After you setup a server, \todo[inline]{ve formálnějších pracech není vhodné používat you. Preferované je "after the server is set up".}
%you can access it via web browser.
%\todo[inline]{není třeba to tu ale dávat do časové souvislosti s tím že někdo setupuje server. Zkrátka system se ovládá pomocí frontendu, který je webový. Ale jelikož využívá API, tak to API lze použít i pro tvorbu nativní mobilní aplikace.}
The system is accessible via web front-end.
The front-end has a support for mobile phones, but it has some drawbacks that will be discussed later.
\todo[inline]{tady bych to propojil s minulou větou, např. The main drawback is that ... } And there is no easy way of managing multiple server instances from one place.

%In this work we're going to \todo[inline]{zase.. vhodnější je In this thesism a solution to both problems is presented....} solve both of those problems at the same time, by making an android application, where you can manage all of your ETL server instances at the same time.
% In this thesis a solution to both problems is presented by making an Android application, where you can manage all of your ETL server instances at the same time.
In this thesis a solution to both problems is presented by making an Android application, where all of the user's \etl{} server instances can be managed at the same time.
The android application will provide nice and smooth mobile experience to \etl{} users, especially to those, who are working with multiple server instances.

\todo[inline]{The rest of this thesis is structured as follows.... a použijte autoref pro odkazování na příslušné sekce}
The rest of this thesis is structured as follows.
Things that are expected from the application will be written down in \autoref{chap:requirementsengineering}.
The written requirements will be compared to existing solutions in \autoref{chap:existingsolutions}.
If no existing solution satisfies our requirements, the application will be designed. \todo[inline]{vynechal bych tuto větu, jelikož kdyby nějaké řešní bylo, neměl byste v práci co dělat, tedy to zde působí úsměvně.}
The application's parts that can be seen and user can interact with alongside with parts that can not be seen, such as the architecture, will be designed in \autoref{chap:design}.
At the end of the \autoref{chap:design}, everything needed for the implementation should be known.
In the \autoref{chap:implementation}, there will be some interesting stuff that happened during programming the app. \todo[inline]{toto je velice neformální. Lépe: popíšete proces implementace řešení.} \todo[inline]{a co ostatní kapitoly?}