LinkedPipes ETL is a system for working with linked data.
\todo[inline]{tady není jasné, co jsou linked data}
\todo[inline]{tady není jasné, co je to "working". dejte příklady - minimálně producing, transforming and consuming} by students and researchers from Prague computer science universities.
\todo[inline]{je taky třeba přidat příslušné reference. Tady konkrétně pomocí BibTeX dle \url{https://dblp.uni-trier.de/rec/conf/iiwas/KlimekS17.html?view=bibtex} - beztak máte dost nedostatečnou sekci bibliografie (jen 2 záznamy)}.
It is an open source project with MIT license \todo[inline]{link}, so anyone can use it.
After you setup a server, \todo[inline]{ve formálnějších pracech není vhodné používat you. Preferované je "after the server is set up".}
you can access it via web browser.
\todo[inline]{není třeba to tu ale dávat do časové souvislosti s tím že někdo setupuje server. Zkrátka system se ovládá pomocí frontendu, který je webový. Ale jelikož využívá API, tak to API lze použít i pro tvorbu nativní mobilní aplikace.}
It has a support for mobile phones, but it has some drawbacks that we will later write about in this work.
But there is no easy way of managing multiple server instances from one place.

In this work we're going to \todo[inline]{zase.. vhodnější je In this thesism a solution to both problems is presented....} solve both of those problems at the same time, by making an android application, where you can manage all of your ETL server instances at the same time.
It will provide nice and smooth mobile experience to ETL users, especially to those, who are working with multiple server instances.

\todo[inline]{The rest of this thesis is structured as follows.... a použijte autoref pro odkazování na příslušné sekce}
We will start by writing down things we expect from the application in the requirements engineering chapter.
Once we know, what we need, we can compare our needs to existing solutions.
When we are sure, that there is no existing solution for our problem, we need to design our application.
We will go through designing parts we can see and users will interact with and then parts that can not be seen, the architecture of the application.
At the end of this chapter, we should know everything we need to know for the implementation.
In the implementation chapter, there will be some interesting stuff that happened during programming the app.