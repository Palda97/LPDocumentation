\ld{} is data that links to other data using URIs.
The URIs identify not only other data objects, but also relations, types or other concepts. \cite{linkeddata}

\etl{} is a system that consumes, transforms and produces \ld{}.
It was created in 2016 by students and researchers from Prague computer science universities \cite{etlpublication}.
It is an open source project \cite{etl} with MIT license, so anyone can use it.
The system is accessible via web front-end.
The front-end has a support for mobile phones, but it has some drawbacks that will be discussed later, with the main drawback being that it is unable to manage multiple server instances from one place.

In this thesis a solution to both problems is presented by making an Android application, where all of the user's \etl{} server instances can be managed at the same time.
The Android application will provide nice and smooth mobile experience to \etl{} users, especially to those, who are working with multiple server instances.

The rest of this thesis is structured as follows.
Requirements for the application will be written down in \autoref{chap:requirementsengineering}.
The written requirements will be compared to existing solutions in \autoref{chap:existingsolutions}.
The application's parts that can be seen and user can interact with will be designed in \autoref{chap:uidesign}.
Parts that can not be seen, such as the architecture, will be designed in \autoref{chap:architecturedesign}.
At the end of the \autoref{chap:architecturedesign}, everything needed for the implementation should be known.
In the \autoref{chap:implementation}, important libraries and some inconspicuous algorithms will be described.
Creation of documentation will be described in \autoref{chap:doc}.
Testing will be described in \autoref{chap:tests}.
