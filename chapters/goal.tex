The goal of this thesis is to pitch an Android client for the \etl{} system as an alternative to opening a web page in a mobile browser.

%In order to do that, we need to go through the whole software developing cycle.
In order to do that, following steps must be fulfilled.
\todo[inline]{tady není třeba vysvětlovat, k čemu která část sw procesu slouží. Spíš to formulujte tak, co je v té které části uděláno.}
%\begin{itemize}
%    \item The goal of requirements engineering is to get together a list of requirements for the application.
%    \item Chapter existing solutions should answer question: "Do we really need to make an application, or is there a ready to use solution for our needs?"
%    \item At the end of design chapter, two things should be clear. How the application will look like and how the code will be layered.
%    \item Result of the implementation layer will be the application.
%    \item After the application is finished, we will write a documentation.
%\end{itemize}
\begin{itemize}
    \item A list of requirements for the application will be put together in the requirements engineering chapter.
    \item Existence of no current solution will be verified in the existing solutions chapter
    \item Application will be designed, including appearance and code layering, in the design chapter.
    \item Result of the implementation chapter will be the application.
    \item Documentation will be created after the application is created.
\end{itemize}