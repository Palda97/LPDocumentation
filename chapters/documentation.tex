This section describes user documentation, developer documentation, system requirements and deployment.

\section{UI Documentation}
A documentation containing screenshots with descriptions (see \autoref{fig:uiDocScreenshot}) will be at the front page of the application's github page, alongside with a link to video tutorials going through previously written use cases.
The project can be found at \url{https://github.com/Palda97/LinkedPipesAndroidClient}.

\begin{figure}\centering
	\includegraphics[width=0.87\textwidth]{pics/uiDocScreenshot.png}
	\caption[UI documentation]{UI documentation}\label{fig:uiDocScreenshot}
\end{figure}

\section{Developer documentation}
In Kotlin, there is a documenting type of comment called KDoc (see \autoref{fig:kdoc}).
With a plugin called Dokka \cite{dokka}, it is possible to construct a website based on the KDoc comments, documenting the application's code.
An example of a part of a class, documented via javadoc can be seen in \autoref{fig:javadoc}.

\begin{figure}\centering
	\includegraphics[width=1\textwidth]{pics/kdoccomment.png}
	\caption[KDoc comment]{KDoc comment}\label{fig:kdoc}
\end{figure}

\begin{figure}\centering
	\includegraphics[width=1\textwidth]{pics/javadoc.png}
	\caption[Example of part of a javadoc page]{Example of part of a javadoc page}\label{fig:javadoc}
\end{figure}

The second part of developer documentation will be written by hand, explaining the internal operations from a greater distance, so other developers can have a more bearable understanding of the application when they follow up on development.
Part of the handwritten developer documentation can be seen in \autoref{fig:developerdocumentation}

\begin{figure}\centering
	\includegraphics[width=1\textwidth]{pics/developerDocumentation.png}
	\caption[Part of the handwritten developer documentation]{Part of the handwritten developer documentation}\label{fig:developerdocumentation}
\end{figure}

Link to developer documentation will be accessible from the application's github page.

\section{System requirements}
The application requires Android 5.0 or greater.
It needs at least 40 MB of space.
An internet connection is needed in order to communicate with servers, but it is not needed for communication with servers on the local network.
Camera is needed for loading server information via QR code.

The system requirements will also be available from the front page of the application's github page.

\section{Deployment}
Google Play will be used for the deployment of this application.
It is an Android application store that is already installed on most Android phones.
Using Google Play is far better than just downloading and installing the app from some storage.
Some of the advantages are simplicity of updating and the option to leave reviews.
The application can be downloaded at \url{https://play.google.com/store/apps/details?id=cz.palda97.lpclient}.