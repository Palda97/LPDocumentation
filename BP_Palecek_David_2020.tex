% arara: pdflatex
% arara: pdflatex
% arara: pdflatex

% options:
% thesis=B bachelor's thesis
% thesis=M master's thesis
% czech thesis in Czech language
% slovak thesis in Slovak language
% english thesis in English language
% hidelinks remove colour boxes around hyperlinks

\documentclass[thesis=B,english]{FITthesis}[2019/12/23]

\usepackage[utf8]{inputenc} % LaTeX source encoded as UTF-8

% \usepackage{amsmath} %advanced maths
% \usepackage{amssymb} %additional math symbols

\usepackage{dirtree} %directory tree visualisation

% % list of acronyms
% \usepackage[acronym,nonumberlist,toc,numberedsection=autolabel]{glossaries}
% \iflanguage{czech}{\renewcommand*{\acronymname}{Seznam pou{\v z}it{\' y}ch zkratek}}{}
% \makeglossaries

\newcommand{\tg}{\mathop{\mathrm{tg}}} %cesky tangens
\newcommand{\cotg}{\mathop{\mathrm{cotg}}} %cesky cotangens

% % % % % % % % % % % % % % % % % % % % % % % % % % % % % % 
% ODTUD DAL VSE ZMENTE
% % % % % % % % % % % % % % % % % % % % % % % % % % % % % % 

\department{Department of Software Engineering}
\title{Mobile client for LinkedPipes ETL on the Android platform}
\authorGN{David} %(křestní) jméno (jména) autora
\authorFN{Paleček} %příjmení autora
\authorWithDegrees{David Paleček} %jméno autora včetně současných akademických titulů
\author{David Paleček} %jméno autora bez akademických titulů
\supervisor{RNDr. Jakub Klímek, Ph.D.}
%\acknowledgements{Doplňte, máte-li komu a za co děkovat. V~opačném případě úplně odstraňte tento příkaz.}
\abstractCS{Tato práce se zabývá tvorbou androidového klienta pro LinkedPipes ETL. Obsahuje všechny stádia vývoje této aplikace.}
\abstractEN{This work deals with the creation of an Android client for LinkedPipes ETL. It contains all stages of development of this application.}
\placeForDeclarationOfAuthenticity{V~Praze}
\declarationOfAuthenticityOption{4} %volba Prohlášení (číslo 1-6)
\keywordsCS{android, klient, linked, pipes, etl}
\keywordsEN{android, client, linked, pipes, etl}
% \website{http://site.example/thesis} %volitelná URL práce, objeví se v tiráži - úplně odstraňte, nemáte-li URL práce
\usepackage{pdfpages}
\begin{document}

% \newacronym{CVUT}{{\v C}VUT}{{\v C}esk{\' e} vysok{\' e} u{\v c}en{\' i} technick{\' e} v Praze}
% \newacronym{FIT}{FIT}{Fakulta informa{\v c}n{\' i}ch technologi{\' i}}

\begin{introduction}
	%sem napište úvod Vaší práce
\end{introduction}

\chapter{Work goal}
Our goal is to pitch an android client for the ETL LinkedPipes system as an alternative to opening a web page in mobile browser.

//TODO

%\include{chapters/templateHelp}

\chapter{Requirements engineering}
Requirements engineering is a process to determine things, the desired system will consist of. This process itself consists of four stages.

\section{Elicitation}
The first faze of requirements engineering is elicitation. It's the faze of gathering information.
For us, it's these two lines:
\begin{itemize}
    %\item The student will design, implement, document and evaluate an Android-based mobile application serving as an alternative client to the current LinkedPipes ETL frontend.
    \item It will be an android-based mobile application serving as an alternative client to the current LinkedPipes ETL frontend.
    \item The application will provide pipeline and execution management and notification capabilities for multiple LinkedPipes ETL instances.
\end{itemize}

\section{Analysis}
Analysis faze is all about making sense from elicitation. It's a systematic approach to elicitation.
%One good method is describing some use cases, as shown in next table:
%\ref{tab:usecases}
%\begin{table}\centering
%	\caption[Use cases]{Use cases}\label{tab:usecases}
%	\begin{tabularx}{\textwidth}{|X|X|}
%        \hline
%        UC-1: Get overview of executions in particular server instance & Enables user to see what pipelines were executed in chronological order from specific server instance. \\ \hline
%        UC-2: Execute specific pipeline & Enables user to execute pipeline of his choice from specific server instance. \\ \hline
%        UC-3: Manage registered server instances & Enables user to register server instance in the application. Application will check, if IP is already registered or if name of the new server instance is already in use, in order to warn user about duplication or name collision that could cause chaos. It also enables user to change IP address of already registered server instance due to type error or network changes. User can also remove registered server instance. \\ \hline
%        UC-4: Manage pipelines & Enables user to manage pipelines in desired server instance. \\ \hline
%        UC-5: Re-execute pipeline from history & Enables user to quickly execute pipeline he stumbles upon while viewing history. \\ \hline
%        UC-6: Delete history & Enables user to delete items from history. \\ \hline
%        UC-7: View execution history & Enables user to view execution history of all the instances at the same time. \\ \hline
%        UC-8: View pipelines & Enables user to view pipelines from all the server instances. \\ \hline
%        UC-9: Be notified on execution finish & User has the option to be notified about execution completion. \\ \hline      
%	\end{tabularx}
%\end{table}
\subsection{Use cases}
One good method of achieving this is by describing some use cases. Each use case is something that user expects from the system. We will also create scenarios for the non trivial ones. Scenario is an array of tasks user should do in order to go through the use case.

\subsubsection*{UC-1: Get overview of executions in particular server instance}
Enables user to see what pipelines were executed in chronological order from specific server instance.
\subsubsection*{UC-2: Execute specific pipeline}
Enables user to execute pipeline of his choice from specific server instance.
\subsubsection*{UC-3: Manage registered server instances}
Enables user to register server instance in the application. Application will check, if IP is already registered or if name of the new server instance is already in use, in order to warn user about duplication or name collision that could cause chaos. It also enables user to change IP address of already registered server instance due to type error or network changes. User can also remove registered server instance.
\subsubsection*{UC-4: Manage pipelines}
Enables user to manage pipelines in desired server instance.
\subsubsection*{UC-5: Re-execute pipeline from history}
Enables user to quickly execute pipeline he stumbles upon while viewing history.
\subsubsection*{UC-6: Delete history}
Enables user to delete items from history.
\subsubsection*{UC-7: View execution history}
Enables user to view execution history of all the instances at the same time.
\subsubsection*{UC-8: View pipelines}
Enables user to view pipelines from all the server instances.
\subsubsection*{UC-9: Be notified on execution finish}
User has the option to be notified about execution completion.

%\subsubsection*{Diagram}
\begin{figure}\centering
	\includegraphics[width=0.9\textwidth]{pics/bc-uc.png}
	\caption[Use cases]{Diagram consisting of use cases}\label{fig:uc}
\end{figure}

%For better perspective, here is a diagram of all these use cases: \ref{fig:uc}

\subsection{Scenarios}
Now we can create some scenarios for the user.

\subsubsection*{SC-1.1: Get overview of executions in particular server instance}
User opens execution history screen and selects what server instance executions he wants to see.
\subsubsection*{SC-2.1: Execute specific pipeline}
User opens pipeline list screen. He then finds the desired pipeline and execute it. Optional: After opening the pipeline screen, user can filter pipelines by the server instance.
\subsubsection*{SC-3.1: Change IP address or name of registered server instance}
User opens settings screen, selects the desired server instance for edit. He then changes the IP address and saves changes.
\subsubsection*{SC-3.2: Register server instance}
User opens settings screen, tells the application he wants to register new server instance and proceeds to enter server instance info and saves it.
\subsubsection*{SC-3.3: Delete registered server instance}
User opens settings screen, views registered server instances and tells the application what server instance he wants to delete, followed by confirmation.
\subsubsection*{SC-4.1: Create pipeline}
User opens pipeline list screen. Then he tells the application he wants to create a new pipeline. He chooses a server instance to which the pipeline will be saved and the screen for editing pipeline will be launched and the user can design a new pipeline here. When he is finished, he will save the pipeline.
\subsubsection*{SC-4.2: Edit pipeline}
User opens pipeline list screen. He then finds the desired pipeline and tells the application he wants to edit it. The screen for editing pipeline will be launched with the selected pipeline loaded so the user can make and save changes here. Optional: After opening the pipeline screen, user can filter pipelines by the server instance.
\subsubsection*{SC-4.3: Delete pipeline}
User opens pipeline list screen. He then finds the desired pipeline and tells the application he wants to delete it. This request is followed by confirmation. Optional: After opening the pipeline screen, user can filter pipelines by the server instance.
\subsubsection*{SC-5.1: Re-execute pipeline from history}
User opens execution history screen. He finds a pipeline and realize he wants to execute it now, so he tells that to the application. Optional: After opening the execution history screen, he can selects what server instance executions he wants to see.
\subsubsection*{SC-6.1: Delete history}
User opens execution history screen. He finds a record and realize he for some reason doesn't want this record in history anymore, so he tells that to the application. Optional: After opening the execution history screen, he can selects what server instance executions he wants to see.
\subsubsection*{SC-9.1: Be notified}
User executes specific pipeline, just like in SC-2.1. Application will notify user about the execution completion. This will happen only if it is allowed in settings.

\section{Specification}
The goal of this section is to produce a structured document consisting of functional requirements.
\subsection*{F-1.0: Settings screen}
Application must have a separate screen for settings. It can be seen in scenarios SC-3.1, SC-3.2, SC-3.3.
\subsection*{F-2.1: View server instance}
List of server instances will be visible from settings screen. It can be seen in scenarios SC-3.1, SC-3.3.
\subsection*{F-2.2: Add server instance}
User must be able to add server instance. It can be seen in scenarios SC-3.2.
\subsection*{F-2.3: Edit server instance}
User must be able to edit already added server instance. It can be seen in scenarios SC-3.1.
\subsection*{F-2.4: Delete server instance}
User must be able to delete already added server instance. Confirmation will be required. It can be seen in scenarios SC-3.3.
\subsection*{F-2.5: Deactivate server instance}
User can deactivate server instance in settings instead of deleting it, so it is possible to activate it again later easily. App will not communicate with deactivated server instances.
\subsection*{F-3.1: Notification after finish}
Application shell create notification on pipeline finish. It can be seen in scenarios SC-9.1.
\subsection*{F-3.2: Notifications in settings}
There will be switch in settings to toggle notifications. It can be seen in scenarios SC-9.1.
\subsection*{F-4.0: Pipeline list screen}
Application must have a separate screen for working with pipelines. Which pipelines will be visible there depends on F-4.7. It can be seen in scenarios SC-2.1, SC-4.1, SC-4.2, SC-4.3.
\subsection*{F-4.1: View pipelines}
List of pipelines will be visible from pipeline list screen. Which pipelines will be visible depends on F-4.7. It can be seen in scenarios SC-2.1, SC-4.2, SC-4.3.
\subsection*{F-4.2: Edit pipeline screen}
Application must have a screen for editing pipelines. It can be seen in scenarios SC-4.1, SC-4.2.
\subsection*{F-4.3: Create pipelines}
User must be able to start an empty edit pipeline screen (F-4.2) from the pipeline list screen (F-4.0). It can be seen in scenarios SC-4.1.
\subsection*{F-4.4: Edit existing pipelines}
User must be able to edit selected pipeline by starting the edit pipeline screen (F-4.2) with the selected activity loaded. It can be seen in scenarios SC-4.2.
\subsection*{F-4.5: Delete pipelines}
User must be able to delete a pipeline of his choice. It can be seen in scenarios SC-4.3.
\subsection*{F-4.6: Execute pipeline}
User must be able to execute selected pipeline. It can be seen in scenarios SC-2.1.
\subsection*{F-4.7: Source for visible pipelines}
User must be able to choose, if he wants to see pipelines from all instances, or just a specific one. It can be seen in scenarios SC-2.1, SC-4.2, SC-4.3.
\subsection*{F-5.0: Execution history screen}
Application must have a separate screen for execution history. History of which server instance will be visible depends on F-5.4 It can be seen in scenarios SC-1.1, SC-5.1, SC-6.1.
\subsection*{F-5.1: View execution history}
List of executions will be visible from the execution history screen. History of which server instance will be visible depends on F-5.4 It can be seen in scenarios SC-1.1, SC-5.1, SC-6.1.
\subsection*{F-5.2: Delete items from history}
User must be able to delete specific item from execution history. Confirmation will be required. It can be seen in scenarios SC-6.1.
\subsection*{F-5.3: Re-execute pipelines from history}
There must be an option to re-execute pipeline from the execution history screen. This action will also make a new record in execution history. It can be seen in scenarios SC-5.1.
\subsection*{F-5.4: Source of visible history}
User must be able to choose, if he wants to see history of all instances, or just a specific one. It can be seen in scenarios SC-1.1, SC-5.1, SC-6.1.
\subsection*{F-6.1: Night mode}
User can have an option in settings to use light or dark theme, or use system default theme (Android 10 and newer).
\subsection*{F-2.6: Ping server}
User can test if the server address is correct

\section{Validation}
In this stage I went through all of previous stages, corrected some typos. I gave everything a second look, to verify everything is somehow testable.

\chapter{Existing solutions}
If there already exists a perfect solution which suits my requirements, it doesn't make sense to create it one more time.

\section{Description of existing solutions}

\subsection{Responsive web app}
It works on any device, not just android devices. Users don't have to download it, which also means they don't have to update it. The responsive web app only works with one server instance. On an android device, it responds slower to screen rotation and animations feel laggy. Even if you don't want to execute anything, just view history or list pipelines, you have to be online. It is also browser dependent.

\section{Summary of existing solutions}

\begin{table}[h]\centering
\caption[Existing solutions]{Features of existing solutions}\label{tab:existingSolutionsTable}
\begin{tabular}{l|l|l}
\hline
Feature & Android App & Web App \\ \hline
Can work with multiple server instances & + & - \\ \hline
Works on any device & - & + \\ \hline
Doesn't need to be downloaded & - & + \\ \hline
Smooth UI & + & - \\ \hline
Can view stuff while offline & + & - \\ \hline
\end{tabular}
\end{table}

\chapter{Design}
//TODO add pictures here from
\url{https://xd.adobe.com/view/4cacfb4f-c6f9-407a-7010-3142a920f0fd-3a9d/}

\section{The three main screens}
From specifications, it's clear that we need at least one screen for execution history, one for pipelines and one for settings.
On the android platform, there are multiple navigation designs.

\subsection{Navigation drawer}
The hamburger icon at the top left and sliding menu from left to the right. If you have 5 or more top level screens, or if you want to have some sort of hierarchical menu, this might be the right choice for you, but that's not our case. We have only three main screens.
Also, with the increasing sizes of mobile phones and most people being right-handed, it's hard to reach the hamburger menu with your right thumb.

\subsection{Bottom navigation}
Three to five icons, optionally with text, at the bottom of the screen. It solves the annoying problem with short thumb, because you no longer need to reach the top left corner with it.
Material design states, that the recommended number of links is three to five. That sounds good. For now, it is one possibility.
% But another recommendation is to make it consistent in a way, that it is visible from all screens.

\subsection{Tabs}
Slidable tabs on top of the screen. Recommendation is at least two screens.
You can click on tab names or you can just slide left or right in order to navigate between the screens. It is another possibility.

\subsection{Conclusion}
The choice is either bottom navigation or tabs. I'm choosing tabs, because I like the idea of sliding between the screens, rather than be forced to only click.

\section{Another screens}

\subsection{Edit server instance}
While registering new server instance or editing an already registered one, the app needs the address for communication and some name for better organising. I will also let the user to add some description on the instance, so he doesn't need to store everything he wants to store about the instance in the server name. There could also be some option to ping the server (F-2.6) to verify the address and a way to cancel the registration/edit.
Because of this all, I want to add another screen just for registering/editing server instance.

\subsection{Edit pipeline screen}
As F-4.2 states, there have to be a screen for editing pipelines. Short description: This is the screen displaying pipeline components and drawing links between them.

\subsection{Edit component screen}
Each component have it's own settings so there is a need for another screen.

%\chapter{Realization}
\chapter{Implementation}
In this chapter we will cover the most important things that are needed to implement the application.
% In this chapter we will cover the most important libraries and algorithms that were used in the implementation process.

\section{Libraries}
While coding the application, there's a need to learn about some libraries. \todo[inline]{why?}

\subsection{Room}
Room makes it easy to work with application's internal database.
For each entity you want to store in database, you define an entity class, which is just a normal kotlin class with some annotations.
For the simplest entity class, you just need to annotate the whole class with \verb|@Entity| and annotate primary key either with \verb|@PrimaryKey| or with multiple primary keys, you can list them as an argument in \verb|@Entity| annotation.
In order to get objects to and from database, you need DAO classes.
The magic is that you declare abstract methods with annotations, optionally with some SQL syntax and Room implements them for you.
Those methods return either the entities, lists of entities or LiveData instances of entities or lists.

\subsection{LiveData}
LiveData is an observable data structure.
Components of view layer can observe those variables, which means that they can react to content changes.
Under the hood, it's a bit more complex.
If content changes multiple times between the screen refresh rate, the view gets only the latest change.
Before observing, the view passes it's own instance of lifecycle, so the LiveData instance knows, when the view is not active or destroyed, so the LiveData instance can delete the reference to this view and stop updating this observer, because of performance and memory leaks.
One of the best things about this library is, that Room can also return LiveData instances, so each time something from the selected group changes, the view knows about it.

\subsection{Retrofit}
Retrofit is a nice android library for making network requests.
The setup is similar to Room DAO classes.
For each API call, you define an abstract method.
It also works very well with OkHttp library, which is different http request library, that among everything supports the basic authentication.

\subsection{Coroutines}
Coroutines are kotlin's way to improve multithreaded programming.
It introduces a new keyword \verb|suspend| that can be written in front of methods.
Methods with the \verb|suspend| keyword can't be called the normal way, but can be called from other \verb|suspend| methods or launched via higher order methods of \verb|CoroutineScope|.

\subsection{WorkManager}
WorkManager library is used to handle services.
Service is a part of an application that can run in background and can be started even when the application is not running.
We will be using this for checking execution statuses.

\subsection{Gson}
Gson makes serializing and deserializing java/kotlin objects really easy.
There are no requirements for the code to change in order to use Gson.
It will be used while transforming incoming network content to objects we can work with.

\subsection{Varvet's QR code scanner}
Google made an API for working with QR codes and Varvet made it even easier.
Varvet made an open source project BarcodeReaderSample and published it under MIT license.

\subsection{DraggableView by hyuwah}
Pipeline's components should be freely movable across the canvas while editing the pipeline.
That can be achieved by making some regular view (button or image) movable.
This is often achieved by a lot of unnecessary code and that's why we will be using this library that should make it easy.

\section{Interesting logic}
This section will describe some of the more complex internal logic.

\subsection{Undo operations}
There are some undo options in planning.
We want the user to be able to undo pipeline deletion or execution deletion.
Just scheduling the sending of the request doesn't do the trick.
We will be creating and storing marks for every scheduled deletion.
So when the application is killed for some reason before the delete request is sent, we can take marks stored in database and send the delete requests at the another start of the application.
Marks can also be used for filtering pipelines and executions that will be showing to the user.
If user undo the deletion, the mark is deleted and the scheduling of the delete request is cancelled.

%\begin{figure}\centering
%    \begin{minipage}[b]{0.45\textwidth}
%    	\includegraphics[width=\textwidth]{pics/undo/delete_action.png}
%    	\caption[Delete action]{Delete action}\label{fig:undoDelete}
%    \end{minipage}
%    \begin{minipage}[b]{0.45\textwidth}
%    	\includegraphics[width=\textwidth]{pics/undo/undo_action.png}
%    	\caption[Undo action]{Undo action}\label{fig:undoUndo}
%    \end{minipage}
%\end{figure}
%\begin{figure}\centering
%    \begin{minipage}[b]{0.45\textwidth}
%    	\includegraphics[width=\textwidth]{pics/undo/db_clean.png}
%    	\caption[Finish interrupted delete actions]{Finish interrupted delete actions}\label{fig:undoDbClean}
%    \end{minipage}
%\end{figure}
\begin{figure}\centering
    \begin{minipage}[b]{0.45\textwidth}
    	\includegraphics[width=\textwidth]{pics/undo/delete_action.png}
    	\caption[Delete action]{Delete action}\label{fig:undoDelete}
    \end{minipage}
    \begin{minipage}[b]{0.45\textwidth}
    	\includegraphics[width=\textwidth]{pics/undo/undo_action.png}
    	\caption[Undo action]{Undo action}\label{fig:undoUndo}
    \end{minipage}
    \begin{minipage}[b]{0.45\textwidth}
    	\includegraphics[width=\textwidth]{pics/undo/db_clean.png}
    	\caption[Finish interrupted delete actions]{Finish interrupted delete actions}\label{fig:undoDbClean}
    \end{minipage}
\end{figure}

\subsection{Execution status}
In order to know, when an execution is finished or cancelled, the application has to periodically check for it.
By looking at the duration length of executions from the public demo server, it is obvious that most of them don't surpass ten seconds.
But looking at executions from one private server, they least for hours and some of them even days.
We will be using this checking strategy:
\begin{itemize}
    \item For the first 10 seconds, check every second.
    \item Next 50 seconds, check every 5\textsuperscript{th} second.
    \item Then check once every hour.
\end{itemize}

\section{Tests}
It is a good practise to write special code that can check whenever parts of the application work as intended.
This code is simply called tests.
This is not only useful to test the application parts when they are written, but these tests could be run in future to ensure that possible code changes didn't broke the functionality of previously written parts.

We will be using libraries Hamcrest and MockK.
Hamcrest will be used for matching lists and MockK for mocking.

\subsection{Local unit tests}
These are tests that require only JVM.
They don't need any part of the android framework, so there's no need to run them on an android device, which makes them fast.
Parsing objects from JSON and back can and will be tested here.

\begin{figure}\centering
	\includegraphics[width=1\textwidth]{pics/coverage.png}
	\caption[Test Coverage]{Picture with local test coverage}\label{fig:coverage}
\end{figure}
We will test the model layer here, but since we don't have access to the database, repository and DAOs will be excluded from these tests.
The coverage, as displayed here: \ref{fig:coverage}, is 89 \% for classes and 85 \% for methods.
70 \% to 80 \% is commonly considered good coverage.

\subsection{Instrumented unit tests}
Instrumented tests require some part of the android framework, so they run in background on an android device.
Previously mentioned library Room requires a part of android framework, so it can be used here.
Repositories will be tested here too.

Unfortunately, the coverage could not be put into operation here.
After researching how coverage can be done on instrumented tests, most of the tutorials are bound to the JaCoCo library, which caused some errors during commissioning attempts.

\section{Documentation}
Looking at a code after some time may feel strange and not very intuitive and that's why documentations exist.
\todo[inline]{nepište to jako příběh. "In this section, we provide the user and developer documentation.}

\subsection{UI Documentation}
No matter how good the UI is, some users may still struggle with it.
For that case, there will be a UI documentation at the front page of the application's github page, alongside with couple of video tutorials going through previously written use cases.

\subsection{Developer documentation}
In kotlin, there are a documenting type of comment called KDoc.
There is also a plugin called Dokka, that can construct a website based on the KDoc comments, documenting application's code.

It's also a good idea to write a piece of documentation by hand, explaining the internal operations from a greater distance, so other developers can have a more bearable understanding of the application when they follow up on development.

Links to both of those will be accessible on the project's github page.

\chapter{Deployment}
Google Play will be used for the deployment of this application.
It is an android application store which is already installed on most of android phones.
Using Google Play is far more better than just downloading and installing the app from some storage.
Some of the advantages are simplicity of updating and the option to leave reviews.
The application will be available to everyone using android version 5.0 (Lollipop) or greater.

\begin{conclusion}
	%sem napište závěr Vaší práce
\end{conclusion}

\bibliographystyle{csn690}
\bibliography{mybibliographyfile}

\appendix

\chapter{Seznam použitých zkratek}
% \printglossaries
\begin{description}
	\item[GUI] Graphical user interface
	\item[XML] Extensible markup language
	\item[API] Application Programming Interface
\item[DAO] Data access object
\item[GUI] Graphical user interface
\item[JSON] JavaScript Object Notation
\item[JVM] Java virtual machine
\item[REST] Representational State Transfer
\item[UI] User interface
\item[URI] Uniform Resource Identifier
\item[XML] Extensible markup language
\end{description}

\chapter{Obsah přiloženého CD}

%upravte podle skutecnosti

\begin{figure}
	\dirtree{%
		.1 readme.txt\DTcomment{stručný popis obsahu CD}.
		.1 exe\DTcomment{adresář se spustitelnou formou implementace}.
		.1 src.
		.2 impl\DTcomment{zdrojové kódy implementace}.
		.2 thesis\DTcomment{zdrojová forma práce ve formátu \LaTeX{}}.
		.1 text\DTcomment{text práce}.
		.2 thesis.pdf\DTcomment{text práce ve formátu PDF}.
		.2 thesis.ps\DTcomment{text práce ve formátu PS}.
	}
\end{figure}

\end{document}
