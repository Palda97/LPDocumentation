% arara: pdflatex
% arara: pdflatex
% arara: pdflatex

% options:
% thesis=B bachelor's thesis
% thesis=M master's thesis
% czech thesis in Czech language
% slovak thesis in Slovak language
% english thesis in English language
% hidelinks remove colour boxes around hyperlinks

\documentclass[thesis=B,english]{FITthesis}[2019/12/23]

\usepackage[utf8]{inputenc} % LaTeX source encoded as UTF-8

% \usepackage{amsmath} %advanced maths
% \usepackage{amssymb} %additional math symbols

\usepackage{dirtree} %directory tree visualisation

\usepackage{todonotes}

% % list of acronyms
% \usepackage[acronym,nonumberlist,toc,numberedsection=autolabel]{glossaries}
% \iflanguage{czech}{\renewcommand*{\acronymname}{Seznam pou{\v z}it{\' y}ch zkratek}}{}
% \makeglossaries

\newcommand{\tg}{\mathop{\mathrm{tg}}} %cesky tangens
\newcommand{\cotg}{\mathop{\mathrm{cotg}}} %cesky cotangens

% % % % % % % % % % % % % % % % % % % % % % % % % % % % % % 
% ODTUD DAL VSE ZMENTE
% % % % % % % % % % % % % % % % % % % % % % % % % % % % % % 

% Auto-Capitalise First Letter
\newcommand{\acfl}[1]{\ifnum\ifhmode\spacefactor\else2000\fi>1000 \uppercase{#1}\else#1\fi}
\newcommand{\etl}{LinkedPipes ETL}
\newcommand{\ld}{\acfl{l}inked data}

\department{Department of Software Engineering}
\title{Mobile client for LinkedPipes ETL on the Android platform}
\authorGN{David} %(křestní) jméno (jména) autora
\authorFN{Paleček} %příjmení autora
\authorWithDegrees{David Paleček} %jméno autora včetně současných akademických titulů
\author{David Paleček} %jméno autora bez akademických titulů
\supervisor{RNDr. Jakub Klímek, Ph.D.}
\acknowledgements{I would like to thank my supervisor RNDr. Jakub Klímek, Ph.D. for his never ending patience and willingness to guide me. Also, I would like to thank my parents for supporting me through my studies.}
%\abstractCS{Tato práce se zabývá tvorbou androidového klienta pro LinkedPipes ETL. Obsahuje všechny stádia vývoje této aplikace. \todo[inline]{Takto abstrakt nestačí. Abstrakt by měl mít několik vět. Přidejte např. co to je LP-ETL, že má jen webové rozhraní, a co konkrétně za stádia práce obsahuje.}}
\abstractCS{
\etl{} je systém pro práci s \ld{}, který má pouze webové rozhraní.
Tato práce se zabývá tvorbou mobilního klienta \etl{} pro platformu Android.
Obsahuje všechny stádia vývoje této aplikace, což jsou sepsání požadavků pro aplikaci, procházení exiistujících řešení, navrhování aplikace, implementace, testování a tvorba dokumentace.
\todo[inline]{Takto abstrakt nestačí. Abstrakt by měl mít několik vět. Přidejte např. co to je LP-ETL, že má jen webové rozhraní, a co konkrétně za stádia práce obsahuje.}
}
% \abstractEN{This work deals with the creation of an Android client for LinkedPipes ETL. It contains all stages of development of this application.}
\abstractEN{
\etl{} is a system for working with \ld{}, which has only a web interface.
This work deals with the creation of the \etl{} mobile client for the Android platform.
It contains all stages of the development of this application, which are writing down the application requirements, the browsing of existing solutions, the designing of application, implementation, testing and the creation of documentation.
}
\placeForDeclarationOfAuthenticity{V~Praze}
\declarationOfAuthenticityOption{5} %volba Prohlášení (číslo 1-6)
\keywordsEN{client creation, android, kotlin, LinkedPipes, ETL}
\keywordsCS{tvorba klienta, android, kotlin, LinkedPipes, ETL}
% \website{http://site.example/thesis} %volitelná URL práce, objeví se v tiráži - úplně odstraňte, nemáte-li URL práce
\usepackage{pdfpages}
\begin{document}

% \newacronym{CVUT}{{\v C}VUT}{{\v C}esk{\' e} vysok{\' e} u{\v c}en{\' i} technick{\' e} v Praze}
% \newacronym{FIT}{FIT}{Fakulta informa{\v c}n{\' i}ch technologi{\' i}}

\begin{introduction}
	%sem napište úvod Vaší práce
	LinkedPipes ETL is a system for working with linked data.
It was created in 2016 by students and researchers from Prague computer science universities.
It is an open source project with MIT license, so anyone can use it.
After you setup a server, you can access it via web browser.
It has a support for mobile phones, but it has some drawbacks that we will later write about in this work.
But there is no easy way of managing multiple server instances from one place.

In this work we're going to solve both of those problems at the same time, by making an android application, where you can manage all of your ETL server instances at the same time.
It will provide nice and smooth mobile experience to ETL users, especially to those, who are working with multiple server instances.

We will start by writing down things we expect from the application in the requirements engineering chapter.
Once we know, what we need, we can compare our needs to existing solutions.
When we are sure, that there is no existing solution for our problem, we need to design our application.
We will go through designing parts we can see and users will interact with and then parts that can not be seen, the architecture of the application.
At the end of this chapter, we should know everything we need to know for the implementation.
In the implementation chapter, there will be some interesting stuff that happened during programming the app.
\end{introduction}

%\chapter{Work goal}
%Our goal is to pitch an android client for the ETL LinkedPipes system as an alternative to opening a web page in mobile browser.

//TODO
%
%%\include{chapters/templateHelp}
%
%\chapter{Requirements engineering}
%Requirements engineering is a process to determine things, the desired system will consist of. This process itself consists of four stages.

\section{Elicitation}
The first faze of requirements engineering is elicitation. It's the faze of gathering information.
For us, it's these two lines:
\begin{itemize}
    %\item The student will design, implement, document and evaluate an Android-based mobile application serving as an alternative client to the current LinkedPipes ETL frontend.
    \item It will be an android-based mobile application serving as an alternative client to the current LinkedPipes ETL frontend.
    \item The application will provide pipeline and execution management and notification capabilities for multiple LinkedPipes ETL instances.
\end{itemize}

\section{Analysis}
Analysis faze is all about making sense from elicitation. It's a systematic approach to elicitation.
%One good method is describing some use cases, as shown in next table:
%\ref{tab:usecases}
%\begin{table}\centering
%	\caption[Use cases]{Use cases}\label{tab:usecases}
%	\begin{tabularx}{\textwidth}{|X|X|}
%        \hline
%        UC-1: Get overview of executions in particular server instance & Enables user to see what pipelines were executed in chronological order from specific server instance. \\ \hline
%        UC-2: Execute specific pipeline & Enables user to execute pipeline of his choice from specific server instance. \\ \hline
%        UC-3: Manage registered server instances & Enables user to register server instance in the application. Application will check, if IP is already registered or if name of the new server instance is already in use, in order to warn user about duplication or name collision that could cause chaos. It also enables user to change IP address of already registered server instance due to type error or network changes. User can also remove registered server instance. \\ \hline
%        UC-4: Manage pipelines & Enables user to manage pipelines in desired server instance. \\ \hline
%        UC-5: Re-execute pipeline from history & Enables user to quickly execute pipeline he stumbles upon while viewing history. \\ \hline
%        UC-6: Delete history & Enables user to delete items from history. \\ \hline
%        UC-7: View execution history & Enables user to view execution history of all the instances at the same time. \\ \hline
%        UC-8: View pipelines & Enables user to view pipelines from all the server instances. \\ \hline
%        UC-9: Be notified on execution finish & User has the option to be notified about execution completion. \\ \hline      
%	\end{tabularx}
%\end{table}
\subsection{Use cases}
One good method of achieving this is by describing some use cases. Each use case is something that user expects from the system. We will also create scenarios for the non trivial ones. Scenario is an array of tasks user should do in order to go through the use case.

\subsubsection*{UC-1: Get overview of executions in particular server instance}
Enables user to see what pipelines were executed in chronological order from specific server instance.
\subsubsection*{UC-2: Execute specific pipeline}
Enables user to execute pipeline of his choice from specific server instance.
\subsubsection*{UC-3: Manage registered server instances}
Enables user to register server instance in the application. Application will check, if IP is already registered or if name of the new server instance is already in use, in order to warn user about duplication or name collision that could cause chaos. It also enables user to change IP address of already registered server instance due to type error or network changes. User can also remove registered server instance.
\subsubsection*{UC-4: Manage pipelines}
Enables user to manage pipelines in desired server instance.
\subsubsection*{UC-5: Re-execute pipeline from history}
Enables user to quickly execute pipeline he stumbles upon while viewing history.
\subsubsection*{UC-6: Delete history}
Enables user to delete items from history.
\subsubsection*{UC-7: View execution history}
Enables user to view execution history of all the instances at the same time.
\subsubsection*{UC-8: View pipelines}
Enables user to view pipelines from all the server instances.
\subsubsection*{UC-9: Be notified on execution finish}
User has the option to be notified about execution completion.

%\subsubsection*{Diagram}
\begin{figure}\centering
	\includegraphics[width=0.9\textwidth]{pics/bc-uc.png}
	\caption[Use cases]{Diagram consisting of use cases}\label{fig:uc}
\end{figure}

%For better perspective, here is a diagram of all these use cases: \ref{fig:uc}

\subsection{Scenarios}
Now we can create some scenarios for the user.

\subsubsection*{SC-1.1: Get overview of executions in particular server instance}
User opens execution history screen and selects what server instance executions he wants to see.
\subsubsection*{SC-2.1: Execute specific pipeline}
User opens pipeline list screen. He then finds the desired pipeline and execute it. Optional: After opening the pipeline screen, user can filter pipelines by the server instance.
\subsubsection*{SC-3.1: Change IP address or name of registered server instance}
User opens settings screen, selects the desired server instance for edit. He then changes the IP address and saves changes.
\subsubsection*{SC-3.2: Register server instance}
User opens settings screen, tells the application he wants to register new server instance and proceeds to enter server instance info and saves it.
\subsubsection*{SC-3.3: Delete registered server instance}
User opens settings screen, views registered server instances and tells the application what server instance he wants to delete, followed by confirmation.
\subsubsection*{SC-4.1: Create pipeline}
User opens pipeline list screen. Then he tells the application he wants to create a new pipeline. He chooses a server instance to which the pipeline will be saved and the screen for editing pipeline will be launched and the user can design a new pipeline here. When he is finished, he will save the pipeline.
\subsubsection*{SC-4.2: Edit pipeline}
User opens pipeline list screen. He then finds the desired pipeline and tells the application he wants to edit it. The screen for editing pipeline will be launched with the selected pipeline loaded so the user can make and save changes here. Optional: After opening the pipeline screen, user can filter pipelines by the server instance.
\subsubsection*{SC-4.3: Delete pipeline}
User opens pipeline list screen. He then finds the desired pipeline and tells the application he wants to delete it. This request is followed by confirmation. Optional: After opening the pipeline screen, user can filter pipelines by the server instance.
\subsubsection*{SC-5.1: Re-execute pipeline from history}
User opens execution history screen. He finds a pipeline and realize he wants to execute it now, so he tells that to the application. Optional: After opening the execution history screen, he can selects what server instance executions he wants to see.
\subsubsection*{SC-6.1: Delete history}
User opens execution history screen. He finds a record and realize he for some reason doesn't want this record in history anymore, so he tells that to the application. Optional: After opening the execution history screen, he can selects what server instance executions he wants to see.
\subsubsection*{SC-9.1: Be notified}
User executes specific pipeline, just like in SC-2.1. Application will notify user about the execution completion. This will happen only if it is allowed in settings.

\section{Specification}
The goal of this section is to produce a structured document consisting of functional requirements.
\subsection*{F-1.0: Settings screen}
Application must have a separate screen for settings. It can be seen in scenarios SC-3.1, SC-3.2, SC-3.3.
\subsection*{F-2.1: View server instance}
List of server instances will be visible from settings screen. It can be seen in scenarios SC-3.1, SC-3.3.
\subsection*{F-2.2: Add server instance}
User must be able to add server instance. It can be seen in scenarios SC-3.2.
\subsection*{F-2.3: Edit server instance}
User must be able to edit already added server instance. It can be seen in scenarios SC-3.1.
\subsection*{F-2.4: Delete server instance}
User must be able to delete already added server instance. Confirmation will be required. It can be seen in scenarios SC-3.3.
\subsection*{F-2.5: Deactivate server instance}
User can deactivate server instance in settings instead of deleting it, so it is possible to activate it again later easily. App will not communicate with deactivated server instances.
\subsection*{F-3.1: Notification after finish}
Application shell create notification on pipeline finish. It can be seen in scenarios SC-9.1.
\subsection*{F-3.2: Notifications in settings}
There will be switch in settings to toggle notifications. It can be seen in scenarios SC-9.1.
\subsection*{F-4.0: Pipeline list screen}
Application must have a separate screen for working with pipelines. Which pipelines will be visible there depends on F-4.7. It can be seen in scenarios SC-2.1, SC-4.1, SC-4.2, SC-4.3.
\subsection*{F-4.1: View pipelines}
List of pipelines will be visible from pipeline list screen. Which pipelines will be visible depends on F-4.7. It can be seen in scenarios SC-2.1, SC-4.2, SC-4.3.
\subsection*{F-4.2: Edit pipeline screen}
Application must have a screen for editing pipelines. It can be seen in scenarios SC-4.1, SC-4.2.
\subsection*{F-4.3: Create pipelines}
User must be able to start an empty edit pipeline screen (F-4.2) from the pipeline list screen (F-4.0). It can be seen in scenarios SC-4.1.
\subsection*{F-4.4: Edit existing pipelines}
User must be able to edit selected pipeline by starting the edit pipeline screen (F-4.2) with the selected activity loaded. It can be seen in scenarios SC-4.2.
\subsection*{F-4.5: Delete pipelines}
User must be able to delete a pipeline of his choice. It can be seen in scenarios SC-4.3.
\subsection*{F-4.6: Execute pipeline}
User must be able to execute selected pipeline. It can be seen in scenarios SC-2.1.
\subsection*{F-4.7: Source for visible pipelines}
User must be able to choose, if he wants to see pipelines from all instances, or just a specific one. It can be seen in scenarios SC-2.1, SC-4.2, SC-4.3.
\subsection*{F-5.0: Execution history screen}
Application must have a separate screen for execution history. History of which server instance will be visible depends on F-5.4 It can be seen in scenarios SC-1.1, SC-5.1, SC-6.1.
\subsection*{F-5.1: View execution history}
List of executions will be visible from the execution history screen. History of which server instance will be visible depends on F-5.4 It can be seen in scenarios SC-1.1, SC-5.1, SC-6.1.
\subsection*{F-5.2: Delete items from history}
User must be able to delete specific item from execution history. Confirmation will be required. It can be seen in scenarios SC-6.1.
\subsection*{F-5.3: Re-execute pipelines from history}
There must be an option to re-execute pipeline from the execution history screen. This action will also make a new record in execution history. It can be seen in scenarios SC-5.1.
\subsection*{F-5.4: Source of visible history}
User must be able to choose, if he wants to see history of all instances, or just a specific one. It can be seen in scenarios SC-1.1, SC-5.1, SC-6.1.
\subsection*{F-6.1: Night mode}
User can have an option in settings to use light or dark theme, or use system default theme (Android 10 and newer).
\subsection*{F-2.6: Ping server}
User can test if the server address is correct

\section{Validation}
In this stage I went through all of previous stages, corrected some typos. I gave everything a second look, to verify everything is somehow testable.
%
%\chapter{Existing solutions}
%If there already exists a perfect solution which suits my requirements, it doesn't make sense to create it one more time.

\section{Description of existing solutions}

\subsection{Responsive web app}
It works on any device, not just android devices. Users don't have to download it, which also means they don't have to update it. The responsive web app only works with one server instance. On an android device, it responds slower to screen rotation and animations feel laggy. Even if you don't want to execute anything, just view history or list pipelines, you have to be online. It is also browser dependent.

\section{Summary of existing solutions}

\begin{table}[h]\centering
\caption[Existing solutions]{Features of existing solutions}\label{tab:existingSolutionsTable}
\begin{tabular}{l|l|l}
\hline
Feature & Android App & Web App \\ \hline
Can work with multiple server instances & + & - \\ \hline
Works on any device & - & + \\ \hline
Doesn't need to be downloaded & - & + \\ \hline
Smooth UI & + & - \\ \hline
Can view stuff while offline & + & - \\ \hline
\end{tabular}
\end{table}
%
%\chapter{Design}
%The appearance of the UI will be described in this chapter.

\section{Design language and UI framework}
Because the application should look decent, some UI guidelines have to be chosen and followed.
These guidelines cover information about colors, shapes and individual components, including their layout.
A set of these guidelines is called design language.
Following design languages are suitable for the Android platform due to the existence of frameworks for this platform, containing themes and components of those languages.

Bootstrap \cite{bootstrap} is a framework for designing web pages, but there also exists a third party library \cite{androidbootstrap} for the Android platform.
Both Microsoft Fluent Design System \cite{fluentui} and Material Design \cite{materialandroid} have their own official libraries available from their representative web pages.

The Bootstrap Android library has not been updated since December 2016 and considering that UI design is always changing and evolving, this library is out of question.
Both Microsoft Fluent Design System and Material Design are being kept up-to-date and are backed by big international companies, which should ensure their stability.
Because our application will be available on the Android platform, which is Google's domain and most of Android phones come with several Google applications pre-installed, Android users are already used to Material Design.

That is why Material Design will be used by our application.

\section{Main screens}
Based on the analysis of the user requirements in \autoref{chap:requirementsengineering}, three screens which cover the functionality of displaying execution history, pipeline list and settings have to be designed.

\section{Main navigation}
On the Android platform, there are multiple navigation designs and they will be described in this section.

\subsection{Navigation drawer}
The hamburger icon at the top left and sliding menu from left to the right is what navigation drawer looks like.
This navigation is suitable for five or more top level screens, or some sort of hierarchical menu \cite{navigationdrawer}.

\subsection{Tabs}
Slidable tabs on top of the screen.
% Recommendation for this type of navigation is having at least two screens.\cite{materialandroid}
Users can click on tab names or just slide left or right in order to navigate between the screens.

\subsection{Bottom navigation}
Bottom navigation consists of icons, usually with text, located at the bottom of the screen.

\section{Conclusion about the main navigation}
The navigation drawer will not be used, because our application does not require five or more main screens nor a hierarchical menu.
Also, with the increasing sizes of mobile phones and most people being right-handed, it is hard to reach the hamburger menu with the right thumb.
There will be lists of items displayed on each of the three main screens.
Those items will be swipeable and having swipeable items on top of swipeable navigation would cause confusion.
Material Design states, that the recommended number of links in the bottom navigation is three to five \cite{bottomnavigation}.
The bottom navigation satisfies our needs and will be used for the main navigation.
The three main screens, containing the bottom navigation, can be seen in \autoref{fig:xdHistory}, \autoref{fig:xdPipelines} and \autoref{fig:xdSettings}.

\begin{figure}\centering
    \begin{minipage}[b]{0.32\textwidth}
    	\includegraphics[width=\textwidth]{pics/xd/Bottom Navigation - executions.png}
    	\caption[History]{History screen design}\label{fig:xdHistory}
    \end{minipage}
    \begin{minipage}[b]{0.32\textwidth}
    	\includegraphics[width=\textwidth]{pics/xd/Bottom Navigation - pipelines.png}
    	\caption[Pipelines]{Pipelines screen design}\label{fig:xdPipelines}
    \end{minipage}
    \begin{minipage}[b]{0.32\textwidth}
    	\includegraphics[width=\textwidth]{pics/xd/Bottom Navigation - settings.png}
    	\caption[Settings]{Settings screen design}\label{fig:xdSettings}
    \end{minipage}
\end{figure}

\section{Lists}
Each of the three main screens will display some sort of list.
For execution screen it is a list of executions, for pipeline screen it is a list of pipelines and for settings screen it is a list of server instances.

All of those lists will have one thing in common and that being the swipe gesture.
When users swipe an item to the left or to the right, the item will be deleted.
This can be seen in \autoref{fig:xdDeletePipeline}.
Users will have the ability to undo this operation for a short period of time.
The undo option can be seen in \autoref{fig:xdUndo}.

Tapping on item from the pipeline screen will open the edit pipeline screen.
Long click on item from execution screen or from pipeline screen will launch the pipeline.

\begin{figure}\centering
    \begin{minipage}[b]{0.32\textwidth}
    	\includegraphics[width=\textwidth]{pics/xd/Bottom Navigation - pipelines – 1.png}
    	\caption[Deleting pipeline]{Deleting pipeline design}\label{fig:xdDeletePipeline}
    \end{minipage}
    \begin{minipage}[b]{0.32\textwidth}
    	\includegraphics[width=\textwidth]{pics/xd/Bottom Navigation - pipelines – 2.png}
    	\caption[Undo option]{Undo option design}\label{fig:xdUndo}
    \end{minipage}
\end{figure}

\section{Edit server instance screen}
While registering new server instance or editing an already registered one, the application needs the address for communication and some name for labeling and better organising.
Users will be able to add a description of the instance, so that there is no pressure to store every information about the instance in the server name.
There could also be an option to ping the server (F-2.6, \autoref{subsec:ping}) to verify the address and a way to cancel the registration/edit.
Because of this, another screen, just for registering/editing server instances, will be added and can be seen in \autoref{fig:xdEditServerInstance}.

\begin{figure}\centering
    \begin{minipage}[b]{0.32\textwidth}
    	\includegraphics[width=\textwidth]{pics/xd/Edit server instance.png}
    	\caption[Edit server instance]{Edit server instance screen design}\label{fig:xdEditServerInstance}
    \end{minipage}
    \begin{minipage}[b]{0.32\textwidth}
    	\includegraphics[width=\textwidth]{pics/xd/Pipeline editor.png}
    	\caption[Edit pipeline screen]{Edit pipeline screen design}\label{fig:xdPipelineEditor}
    \end{minipage}
\end{figure}

\section{Edit pipeline screen}
According to the F-4.3 requirement, described in \autoref{subsec:editpipelinescreen}, there have to be a screen for editing pipelines.
This screen will be displaying pipeline components and drawing links between them.
The preliminary design of this screen can be seen in \autoref{fig:xdPipelineEditor}

% \begin{figure}\centering
%     \begin{minipage}[b]{0.32\textwidth}
%     	\includegraphics[width=\textwidth]{pics/xd/Create new pipeline.png}
%     	\caption[Create new pipeline dialog]{Create new pipeline dialog design}\label{fig:xdCreateNewPipelineDialog}
%     \end{minipage}
%     \begin{minipage}[b]{0.32\textwidth}
%     	\includegraphics[width=\textwidth]{pics/xd/Pipeline editor.png}
%     	\caption[Edit pipeline screen]{Edit pipeline screen design}\label{fig:xdPipelineEditor}
%     \end{minipage}
% \end{figure}

\section{Edit component screen}
This screen has to be created, because each pipeline's component has it's own settings.
In \autoref{fig:xdEditComponent1} and \autoref{fig:xdEditComponent2} is the preliminary design of this screen.

\begin{figure}\centering
    \begin{minipage}[b]{0.32\textwidth}
    	\includegraphics[width=\textwidth]{pics/xd/Edit component - configuration.png}
    	\caption[Edit component screen 1]{Edit component screen 1 design}\label{fig:xdEditComponent1}
    \end{minipage}
    \begin{minipage}[b]{0.32\textwidth}
    	\includegraphics[width=\textwidth]{pics/xd/Edit component - io.png}
    	\caption[Edit component screen 2]{Edit component screen 2 design}\label{fig:xdEditComponent2}
    \end{minipage}
\end{figure}

\section{Notifications}
According to the F-3.1 requirement, described in \autoref{subsec:notifications}, notifications need to be implemented.
The preview of notifications can be seen in \autoref{fig:xdNotifications}.

\begin{figure}\centering
    \begin{minipage}[b]{0.7\textwidth}
    	\includegraphics[width=\textwidth]{pics/xd/Notifications.png}
    	\caption[Notifications]{Notification design}\label{fig:xdNotifications}
    \end{minipage}
\end{figure}
%% https://www.youtube.com/watch?v=ugpC98LcNqA
% https://www.youtube.com/watch?v=QrbhPcbZv0I

\section{Software architecture patterns}
Every non trivial software project should follow some architecture design so it always remains maintainable and expandable as much easily as possible.
The project should be modularized, so that any part could be replaced without the need of rewriting the entire project.
And for this reason, there exist architecture patterns.

\subsection{Basic patterns}
% We will go through three of them and explain the conclusion.

% All of these three patterns have one thing in common.
% They structure code into three main layers.
% The names of those layers are present in these pattern's names.

The three basic software architecture patterns will be described here.
The conclusion on which one to pick for our application will be made afterwards.

All of these patterns have one thing in common, that being structuring code into three main layers.
The names of those layers are present in these pattern's names.

\subsubsection{MVC}

\begin{figure}\centering
	\includegraphics[width=1\textwidth]{pics/patterns/bc-mvc2.png}
	\caption[MVC]{Model View Controller}\label{fig:mvc}
\end{figure}

\textbf{Model View Controller}.
View is responsible for displaying data, controller is responsible for getting the user input and model for storing and serving data.
The controller takes user input, it updates the model and then tells the view to update itself. \cite{droidcon}

On the Android platform, the part of application responsible for displaying data is also responsible for dealing with user input.
The pattern can be modified the way that the view gets the user input, sends it to the controller, the controller updated the model and then informs the view to update itself, based on the data from model. \cite{droidcon}

So both view and controller know the model, the view knows controller and the controller knows the view.
That's high consistency and that is a thing to be avoided, especially in the Android world, where forgetting to remove a link can and will cause memory leaks.

And what if the data should be somehow transformed for presentation?
What part should do this presentation transformation, also known as UI logic?
This logic should not be pushed to the view, because it's purpose is only display the given data.
The controller and the model can not posses it neither, because the controller doesn't supply any data to the view and model is responsible for data storing and serving and there is no reason why it should know how to display data.

\subsubsection{MVP}

\begin{figure}\centering
	\includegraphics[width=0.7\textwidth]{pics/patterns/bc-mvp.png}
	\caption[MVP]{Model View Presenter}\label{fig:mvp}
\end{figure}

Following the MVC and the UI logic problem.
What if the view doesn't communicate with the model, but only with controller and the controller was also responsible for taking data from the model and supplying them to the view.
The UI logic could be put in this new controller. This new controller will be called presenter, instead of new controller.
And that's what \textbf{Model View Presenter} is. \cite{droidcon}

But that means the presenter still holds a link to the view and some repetitive code would have to be written because of this. \cite{mvp}
Also, the presenter should not know what parts of the view should be updated after the data changes, because displaying data is not his job.

\subsubsection{MVVM}
\label{subsec:mvvm}

\begin{figure}\centering
	\includegraphics[width=0.7\textwidth]{pics/patterns/bc-mvvm.png}
	\caption[MVVM]{Model View ViewModel}\label{fig:mvvm}
\end{figure}

\textbf{Model View Viewmodel}.
When following the MVP, the presenter knows the view and is executing it's methods when it needs to be updated.
The presenter will no longer know it's view.
Instead it will provide some kind of stream and the view will be able observe the stream, so it can display the data whenever the data changes.
This new presenter will be called viewmodel.

The view now knows the viewmodel, can call it's methods on user input and observe it's data streams.
The viewmodel knows model, observes it's data streams and omits them to the view.
The model doesn't know the view or the viewmodel.

\subsection{Conclusion about picking the pattern}
MVVM suits applications for the Android platform the best.
Google even made some libraries to support MVVM.

\section{Main layers and libraries}
The three main layers of the MVVM will be described here, including a new fourth layer separated from the model.
Some libraries that will be used inside of those layers will also be described here.

\subsection{View}
As stated before, the view is the layer responsible for displaying data.
The way it's implemented is through a combination of standard Kotlin code and XML.
The XML is mostly used to forming the display structure and Kotlin is mostly used to determine what data the view should display, how to update itself and what to do with user input.

In order to work this way, XML and Kotlin have to be somehow linked together.
A library called Data Binding will be used for this purpose.
Compared to other solutions, like Kotlin synthetics, Data Binding offers more features and has a more robust outlook.

\subsection{Viewmodel}
Google has made some architecture components and one of them is exactly for viewmodel.
There is no better way to represent this layer, so this component will be used in our application.

\subsection{Repository}
Repository will represent the part of model responsible for caching and maintaining data stored in the application's local database through communication with the following layer.

\subsection{DAO and network IO}
This will be the fourth layer responsible for storing and loading data and communicating over the internet.

Storing and loading data is a common problem with already existing solutions.
These solutions are called persistence libraries.
One persistence library will be chosen for our application, so the data maintaining doesn't have to be implemented again.
The three popular persistence libraries for the Android platform are Room\cite{room}, Realm\cite{realm} and ObjectBox\cite{objectbox}.
Out of these three, only Room offers the usage of custom written SQL commands, thus providing more flexibility and will be used for our application.

Communicating over the internet is also a common problem, so an existing library will be chosen.
Between commonly used HTTP libraries, not specialised for images, belong Volley\cite{volley} and Retrofit\cite{retrofit}.
Volley is more of a HTTP client and Retrofit makes it very easy to adapt to REST APIs.
Our application will be communicating with the \etl 's REST API and Retrofit is a more fitting solution.

\subsection{LiveData}
Previously in \autoref{subsec:mvvm}, ``some kind of stream'' was mentioned.
For this, an observable data structure is needed and that's what LiveData\cite{livedata} is.
Compared to RxJava\cite{rxjava}, another stream like data structure, LiveData misses some features like working on a background thread and is a bit more complex to use, but LiveData is lifecycle-aware.
Views have something called lifecycle.
Once they're not on screen, all links to them have to removed, otherwise memory leaks occur.
LiveData objects respect these lifecycles, so no additional repetitive code has to be written.
Also, LiveData objects are not really streams, because they do nothing when there is no observer.
If content changes multiple times between  the  screen  refresh  rate,  the  view  gets  only  the  latest  change.
LiveData will be used to propagate data all the way from model, through viewmodel, to view in the MVVM architecture.\todo[inline]{why?}

\section{Conclusion}

\begin{figure}\centering
	\includegraphics[width=1\textwidth]{pics/bc-architecture.png}
	\caption[Architecture]{Diagram consisting of MVVM and server instances}\label{fig:architecture}
\end{figure}

Repository will communicate with the DAO and Network layer in order to download, store and load data.
Viewmodel will communicate with repository and offer data ready to display to the view layer.
View layer will display data and react to user input by forwarding it to the viewmodel.
Passing data to the view layer will be handed using LiveData.
The summary can be seen in \autoref{fig:architecture}\todo[inline]{use autoref}
%
%%\chapter{Realization}
%\chapter{Implementation}
%In this chapter we will cover the most important things that are needed to implement the application.
% In this chapter we will cover the most important libraries and algorithms that were used in the implementation process.

\section{Libraries}
While coding the application, there's a need to learn about some libraries. \todo[inline]{why?}

\subsection{Room}
Room makes it easy to work with application's internal database.
For each entity you want to store in database, you define an entity class, which is just a normal kotlin class with some annotations.
For the simplest entity class, you just need to annotate the whole class with \verb|@Entity| and annotate primary key either with \verb|@PrimaryKey| or with multiple primary keys, you can list them as an argument in \verb|@Entity| annotation.
In order to get objects to and from database, you need DAO classes.
The magic is that you declare abstract methods with annotations, optionally with some SQL syntax and Room implements them for you.
Those methods return either the entities, lists of entities or LiveData instances of entities or lists.

\subsection{LiveData}
LiveData is an observable data structure.
Components of view layer can observe those variables, which means that they can react to content changes.
Under the hood, it's a bit more complex.
If content changes multiple times between the screen refresh rate, the view gets only the latest change.
Before observing, the view passes it's own instance of lifecycle, so the LiveData instance knows, when the view is not active or destroyed, so the LiveData instance can delete the reference to this view and stop updating this observer, because of performance and memory leaks.
One of the best things about this library is, that Room can also return LiveData instances, so each time something from the selected group changes, the view knows about it.

\subsection{Retrofit}
Retrofit is a nice android library for making network requests.
The setup is similar to Room DAO classes.
For each API call, you define an abstract method.
It also works very well with OkHttp library, which is different http request library, that among everything supports the basic authentication.

\subsection{Coroutines}
Coroutines are kotlin's way to improve multithreaded programming.
It introduces a new keyword \verb|suspend| that can be written in front of methods.
Methods with the \verb|suspend| keyword can't be called the normal way, but can be called from other \verb|suspend| methods or launched via higher order methods of \verb|CoroutineScope|.

\subsection{WorkManager}
WorkManager library is used to handle services.
Service is a part of an application that can run in background and can be started even when the application is not running.
We will be using this for checking execution statuses.

\subsection{Gson}
Gson makes serializing and deserializing java/kotlin objects really easy.
There are no requirements for the code to change in order to use Gson.
It will be used while transforming incoming network content to objects we can work with.

\subsection{Varvet's QR code scanner}
Google made an API for working with QR codes and Varvet made it even easier.
Varvet made an open source project BarcodeReaderSample and published it under MIT license.

\subsection{DraggableView by hyuwah}
Pipeline's components should be freely movable across the canvas while editing the pipeline.
That can be achieved by making some regular view (button or image) movable.
This is often achieved by a lot of unnecessary code and that's why we will be using this library that should make it easy.

\section{Interesting logic}
This section will describe some of the more complex internal logic.

\subsection{Undo operations}
There are some undo options in planning.
We want the user to be able to undo pipeline deletion or execution deletion.
Just scheduling the sending of the request doesn't do the trick.
We will be creating and storing marks for every scheduled deletion.
So when the application is killed for some reason before the delete request is sent, we can take marks stored in database and send the delete requests at the another start of the application.
Marks can also be used for filtering pipelines and executions that will be showing to the user.
If user undo the deletion, the mark is deleted and the scheduling of the delete request is cancelled.

%\begin{figure}\centering
%    \begin{minipage}[b]{0.45\textwidth}
%    	\includegraphics[width=\textwidth]{pics/undo/delete_action.png}
%    	\caption[Delete action]{Delete action}\label{fig:undoDelete}
%    \end{minipage}
%    \begin{minipage}[b]{0.45\textwidth}
%    	\includegraphics[width=\textwidth]{pics/undo/undo_action.png}
%    	\caption[Undo action]{Undo action}\label{fig:undoUndo}
%    \end{minipage}
%\end{figure}
%\begin{figure}\centering
%    \begin{minipage}[b]{0.45\textwidth}
%    	\includegraphics[width=\textwidth]{pics/undo/db_clean.png}
%    	\caption[Finish interrupted delete actions]{Finish interrupted delete actions}\label{fig:undoDbClean}
%    \end{minipage}
%\end{figure}
\begin{figure}\centering
    \begin{minipage}[b]{0.45\textwidth}
    	\includegraphics[width=\textwidth]{pics/undo/delete_action.png}
    	\caption[Delete action]{Delete action}\label{fig:undoDelete}
    \end{minipage}
    \begin{minipage}[b]{0.45\textwidth}
    	\includegraphics[width=\textwidth]{pics/undo/undo_action.png}
    	\caption[Undo action]{Undo action}\label{fig:undoUndo}
    \end{minipage}
    \begin{minipage}[b]{0.45\textwidth}
    	\includegraphics[width=\textwidth]{pics/undo/db_clean.png}
    	\caption[Finish interrupted delete actions]{Finish interrupted delete actions}\label{fig:undoDbClean}
    \end{minipage}
\end{figure}

\subsection{Execution status}
In order to know, when an execution is finished or cancelled, the application has to periodically check for it.
By looking at the duration length of executions from the public demo server, it is obvious that most of them don't surpass ten seconds.
But looking at executions from one private server, they least for hours and some of them even days.
We will be using this checking strategy:
\begin{itemize}
    \item For the first 10 seconds, check every second.
    \item Next 50 seconds, check every 5\textsuperscript{th} second.
    \item Then check once every hour.
\end{itemize}

\section{Tests}
It is a good practise to write special code that can check whenever parts of the application work as intended.
This code is simply called tests.
This is not only useful to test the application parts when they are written, but these tests could be run in future to ensure that possible code changes didn't broke the functionality of previously written parts.

We will be using libraries Hamcrest and MockK.
Hamcrest will be used for matching lists and MockK for mocking.

\subsection{Local unit tests}
These are tests that require only JVM.
They don't need any part of the android framework, so there's no need to run them on an android device, which makes them fast.
Parsing objects from JSON and back can and will be tested here.

\begin{figure}\centering
	\includegraphics[width=1\textwidth]{pics/coverage.png}
	\caption[Test Coverage]{Picture with local test coverage}\label{fig:coverage}
\end{figure}
We will test the model layer here, but since we don't have access to the database, repository and DAOs will be excluded from these tests.
The coverage, as displayed here: \ref{fig:coverage}, is 89 \% for classes and 85 \% for methods.
70 \% to 80 \% is commonly considered good coverage.

\subsection{Instrumented unit tests}
Instrumented tests require some part of the android framework, so they run in background on an android device.
Previously mentioned library Room requires a part of android framework, so it can be used here.
Repositories will be tested here too.

Unfortunately, the coverage could not be put into operation here.
After researching how coverage can be done on instrumented tests, most of the tutorials are bound to the JaCoCo library, which caused some errors during commissioning attempts.

\section{Documentation}
Looking at a code after some time may feel strange and not very intuitive and that's why documentations exist.
\todo[inline]{nepište to jako příběh. "In this section, we provide the user and developer documentation.}

\subsection{UI Documentation}
No matter how good the UI is, some users may still struggle with it.
For that case, there will be a UI documentation at the front page of the application's github page, alongside with couple of video tutorials going through previously written use cases.

\subsection{Developer documentation}
In kotlin, there are a documenting type of comment called KDoc.
There is also a plugin called Dokka, that can construct a website based on the KDoc comments, documenting application's code.

It's also a good idea to write a piece of documentation by hand, explaining the internal operations from a greater distance, so other developers can have a more bearable understanding of the application when they follow up on development.

Links to both of those will be accessible on the project's github page.
%
%\chapter{Deployment}
%Google Play will be used for the deployment of this application.
It is an android application store which is already installed on most of android phones.
Using Google Play is far more better than just downloading and installing the app from some storage.
Some of the advantages are simplicity of updating and the option to leave reviews.
The application will be available to everyone using android version 5.0 (Lollipop) or greater.

\chapter*{Work goal}
\addcontentsline{toc}{chapter}{Work goal}
Our goal is to pitch an android client for the ETL LinkedPipes system as an alternative to opening a web page in mobile browser.

//TODO

%\chapter{Literary research}
%\todo[inline]{tady té kapitole nerozumím. Bibliografie je sekce na konci práce.}
%%\begin{itemize}
%    \item Material design
%    \item Architecture patterns
%    \item Android libraries such as Room, LiveData and Work Manager
%    \item LinkedPipes ETL REST API documentation.
%\end{itemize}

\begin{enumerate}
    \item Material Design [online], Google LLC, [cit. 2020-12-06], Available from: https://material.io
%    \item MUNTENESCU, Florina and Touchlab, Droidcon NYC 2016 - A Journey Through MV Wonderland (updated), Youtube [video], Youtube LLC, 2016, [cit. 2020-12-06], Available from:
    
    https://www.youtube.com/watch?v=QrbhPcbZv0I
    \item Android Developers [online], Google LLC, [cit. 2020-12-06], Available from: https://developer.android.com/
    \item Petr ŠKODA, Jakub KLÍMEK, Martin NEČASKÝ, LinkedPipes ETL REST API. In: Github [online], GitHub Inc., 28 Oct 2017, [cit. 2020-12-06], Available from: https://github.com/linkedpipes/etl/wiki/LinkedPipes-ETL-REST-API
\end{enumerate}

%\input{chapters/testing}
%\input{chapters/templateHelp}

\chapter{Requirements engineering}
\label{chap:requirementsengineering}
Requirements engineering is a process to determine things, the desired system will consist of. This process itself consists of four stages.

\section{Elicitation}
The first faze of requirements engineering is elicitation. It's the faze of gathering information.
For us, it's these two lines:
\begin{itemize}
    %\item The student will design, implement, document and evaluate an Android-based mobile application serving as an alternative client to the current LinkedPipes ETL frontend.
    \item It will be an android-based mobile application serving as an alternative client to the current LinkedPipes ETL frontend.
    \item The application will provide pipeline and execution management and notification capabilities for multiple LinkedPipes ETL instances.
\end{itemize}

\section{Analysis}
Analysis faze is all about making sense from elicitation. It's a systematic approach to elicitation.
%One good method is describing some use cases, as shown in next table:
%\ref{tab:usecases}
%\begin{table}\centering
%	\caption[Use cases]{Use cases}\label{tab:usecases}
%	\begin{tabularx}{\textwidth}{|X|X|}
%        \hline
%        UC-1: Get overview of executions in particular server instance & Enables user to see what pipelines were executed in chronological order from specific server instance. \\ \hline
%        UC-2: Execute specific pipeline & Enables user to execute pipeline of his choice from specific server instance. \\ \hline
%        UC-3: Manage registered server instances & Enables user to register server instance in the application. Application will check, if IP is already registered or if name of the new server instance is already in use, in order to warn user about duplication or name collision that could cause chaos. It also enables user to change IP address of already registered server instance due to type error or network changes. User can also remove registered server instance. \\ \hline
%        UC-4: Manage pipelines & Enables user to manage pipelines in desired server instance. \\ \hline
%        UC-5: Re-execute pipeline from history & Enables user to quickly execute pipeline he stumbles upon while viewing history. \\ \hline
%        UC-6: Delete history & Enables user to delete items from history. \\ \hline
%        UC-7: View execution history & Enables user to view execution history of all the instances at the same time. \\ \hline
%        UC-8: View pipelines & Enables user to view pipelines from all the server instances. \\ \hline
%        UC-9: Be notified on execution finish & User has the option to be notified about execution completion. \\ \hline      
%	\end{tabularx}
%\end{table}
\subsection{Use cases}
One good method of achieving this is by describing some use cases. Each use case is something that user expects from the system. We will also create scenarios for the non trivial ones. Scenario is an array of tasks user should do in order to go through the use case.

\subsubsection*{UC-1: Get overview of executions in particular server instance}
Enables user to see what pipelines were executed in chronological order from specific server instance.
\subsubsection*{UC-2: Execute specific pipeline}
Enables user to execute pipeline of his choice from specific server instance.
\subsubsection*{UC-3: Manage registered server instances}
Enables user to register server instance in the application. Application will check, if IP is already registered or if name of the new server instance is already in use, in order to warn user about duplication or name collision that could cause chaos. It also enables user to change IP address of already registered server instance due to type error or network changes. User can also remove registered server instance.
\subsubsection*{UC-4: Manage pipelines}
Enables user to manage pipelines in desired server instance.
\subsubsection*{UC-5: Re-execute pipeline from history}
Enables user to quickly execute pipeline he stumbles upon while viewing history.
\subsubsection*{UC-6: Delete history}
Enables user to delete items from history.
\subsubsection*{UC-7: View execution history}
Enables user to view execution history of all the instances at the same time.
\subsubsection*{UC-8: View pipelines}
Enables user to view pipelines from all the server instances.
\subsubsection*{UC-9: Be notified on execution finish}
User has the option to be notified about execution completion.

%\subsubsection*{Diagram}
\begin{figure}\centering
	\includegraphics[width=0.9\textwidth]{pics/bc-uc.png}
	\caption[Use cases]{Diagram consisting of use cases}\label{fig:uc}
\end{figure}

%For better perspective, here is a diagram of all these use cases: \ref{fig:uc}

\subsection{Scenarios}
Now we can create some scenarios for the user.

\subsubsection*{SC-1.1: Get overview of executions in particular server instance}
User opens execution history screen and selects what server instance executions he wants to see.
\subsubsection*{SC-2.1: Execute specific pipeline}
User opens pipeline list screen. He then finds the desired pipeline and execute it. Optional: After opening the pipeline screen, user can filter pipelines by the server instance.
\subsubsection*{SC-3.1: Change IP address or name of registered server instance}
User opens settings screen, selects the desired server instance for edit. He then changes the IP address and saves changes.
\subsubsection*{SC-3.2: Register server instance}
User opens settings screen, tells the application he wants to register new server instance and proceeds to enter server instance info and saves it.
\subsubsection*{SC-3.3: Delete registered server instance}
User opens settings screen, views registered server instances and tells the application what server instance he wants to delete, followed by confirmation.
\subsubsection*{SC-4.1: Create pipeline}
User opens pipeline list screen. Then he tells the application he wants to create a new pipeline. He chooses a server instance to which the pipeline will be saved and the screen for editing pipeline will be launched and the user can design a new pipeline here. When he is finished, he will save the pipeline.
\subsubsection*{SC-4.2: Edit pipeline}
User opens pipeline list screen. He then finds the desired pipeline and tells the application he wants to edit it. The screen for editing pipeline will be launched with the selected pipeline loaded so the user can make and save changes here. Optional: After opening the pipeline screen, user can filter pipelines by the server instance.
\subsubsection*{SC-4.3: Delete pipeline}
User opens pipeline list screen. He then finds the desired pipeline and tells the application he wants to delete it. This request is followed by confirmation. Optional: After opening the pipeline screen, user can filter pipelines by the server instance.
\subsubsection*{SC-5.1: Re-execute pipeline from history}
User opens execution history screen. He finds a pipeline and realize he wants to execute it now, so he tells that to the application. Optional: After opening the execution history screen, he can selects what server instance executions he wants to see.
\subsubsection*{SC-6.1: Delete history}
User opens execution history screen. He finds a record and realize he for some reason doesn't want this record in history anymore, so he tells that to the application. Optional: After opening the execution history screen, he can selects what server instance executions he wants to see.
\subsubsection*{SC-9.1: Be notified}
User executes specific pipeline, just like in SC-2.1. Application will notify user about the execution completion. This will happen only if it is allowed in settings.

\section{Specification}
The goal of this section is to produce a structured document consisting of functional requirements.
\subsection*{F-1.0: Settings screen}
Application must have a separate screen for settings. It can be seen in scenarios SC-3.1, SC-3.2, SC-3.3.
\subsection*{F-2.1: View server instance}
List of server instances will be visible from settings screen. It can be seen in scenarios SC-3.1, SC-3.3.
\subsection*{F-2.2: Add server instance}
User must be able to add server instance. It can be seen in scenarios SC-3.2.
\subsection*{F-2.3: Edit server instance}
User must be able to edit already added server instance. It can be seen in scenarios SC-3.1.
\subsection*{F-2.4: Delete server instance}
User must be able to delete already added server instance. Confirmation will be required. It can be seen in scenarios SC-3.3.
\subsection*{F-2.5: Deactivate server instance}
User can deactivate server instance in settings instead of deleting it, so it is possible to activate it again later easily. App will not communicate with deactivated server instances.
\subsection*{F-3.1: Notification after finish}
Application shell create notification on pipeline finish. It can be seen in scenarios SC-9.1.
\subsection*{F-3.2: Notifications in settings}
There will be switch in settings to toggle notifications. It can be seen in scenarios SC-9.1.
\subsection*{F-4.0: Pipeline list screen}
Application must have a separate screen for working with pipelines. Which pipelines will be visible there depends on F-4.7. It can be seen in scenarios SC-2.1, SC-4.1, SC-4.2, SC-4.3.
\subsection*{F-4.1: View pipelines}
List of pipelines will be visible from pipeline list screen. Which pipelines will be visible depends on F-4.7. It can be seen in scenarios SC-2.1, SC-4.2, SC-4.3.
\subsection*{F-4.2: Edit pipeline screen}
Application must have a screen for editing pipelines. It can be seen in scenarios SC-4.1, SC-4.2.
\subsection*{F-4.3: Create pipelines}
User must be able to start an empty edit pipeline screen (F-4.2) from the pipeline list screen (F-4.0). It can be seen in scenarios SC-4.1.
\subsection*{F-4.4: Edit existing pipelines}
User must be able to edit selected pipeline by starting the edit pipeline screen (F-4.2) with the selected activity loaded. It can be seen in scenarios SC-4.2.
\subsection*{F-4.5: Delete pipelines}
User must be able to delete a pipeline of his choice. It can be seen in scenarios SC-4.3.
\subsection*{F-4.6: Execute pipeline}
User must be able to execute selected pipeline. It can be seen in scenarios SC-2.1.
\subsection*{F-4.7: Source for visible pipelines}
User must be able to choose, if he wants to see pipelines from all instances, or just a specific one. It can be seen in scenarios SC-2.1, SC-4.2, SC-4.3.
\subsection*{F-5.0: Execution history screen}
Application must have a separate screen for execution history. History of which server instance will be visible depends on F-5.4 It can be seen in scenarios SC-1.1, SC-5.1, SC-6.1.
\subsection*{F-5.1: View execution history}
List of executions will be visible from the execution history screen. History of which server instance will be visible depends on F-5.4 It can be seen in scenarios SC-1.1, SC-5.1, SC-6.1.
\subsection*{F-5.2: Delete items from history}
User must be able to delete specific item from execution history. Confirmation will be required. It can be seen in scenarios SC-6.1.
\subsection*{F-5.3: Re-execute pipelines from history}
There must be an option to re-execute pipeline from the execution history screen. This action will also make a new record in execution history. It can be seen in scenarios SC-5.1.
\subsection*{F-5.4: Source of visible history}
User must be able to choose, if he wants to see history of all instances, or just a specific one. It can be seen in scenarios SC-1.1, SC-5.1, SC-6.1.
\subsection*{F-6.1: Night mode}
User can have an option in settings to use light or dark theme, or use system default theme (Android 10 and newer).
\subsection*{F-2.6: Ping server}
User can test if the server address is correct

\section{Validation}
In this stage I went through all of previous stages, corrected some typos. I gave everything a second look, to verify everything is somehow testable.

\chapter{Existing solutions}
\label{chap:existingsolutions}
If there already exists a perfect solution which suits my requirements, it doesn't make sense to create it one more time.

\section{Description of existing solutions}

\subsection{Responsive web app}
It works on any device, not just android devices. Users don't have to download it, which also means they don't have to update it. The responsive web app only works with one server instance. On an android device, it responds slower to screen rotation and animations feel laggy. Even if you don't want to execute anything, just view history or list pipelines, you have to be online. It is also browser dependent.

\section{Summary of existing solutions}

\begin{table}[h]\centering
\caption[Existing solutions]{Features of existing solutions}\label{tab:existingSolutionsTable}
\begin{tabular}{l|l|l}
\hline
Feature & Android App & Web App \\ \hline
Can work with multiple server instances & + & - \\ \hline
Works on any device & - & + \\ \hline
Doesn't need to be downloaded & - & + \\ \hline
Smooth UI & + & - \\ \hline
Can view stuff while offline & + & - \\ \hline
\end{tabular}
\end{table}

\chapter{Design}
\label{chap:design}
The appearance of the UI will be described in this chapter.

\section{Design language and UI framework}
Because the application should look decent, some UI guidelines have to be chosen and followed.
These guidelines cover information about colors, shapes and individual components, including their layout.
A set of these guidelines is called design language.
Following design languages are suitable for the Android platform due to the existence of frameworks for this platform, containing themes and components of those languages.

Bootstrap \cite{bootstrap} is a framework for designing web pages, but there also exists a third party library \cite{androidbootstrap} for the Android platform.
Both Microsoft Fluent Design System \cite{fluentui} and Material Design \cite{materialandroid} have their own official libraries available from their representative web pages.

The Bootstrap Android library has not been updated since December 2016 and considering that UI design is always changing and evolving, this library is out of question.
Both Microsoft Fluent Design System and Material Design are being kept up-to-date and are backed by big international companies, which should ensure their stability.
Because our application will be available on the Android platform, which is Google's domain and most of Android phones come with several Google applications pre-installed, Android users are already used to Material Design.

That is why Material Design will be used by our application.

\section{Main screens}
Based on the analysis of the user requirements in \autoref{chap:requirementsengineering}, three screens which cover the functionality of displaying execution history, pipeline list and settings have to be designed.

\section{Main navigation}
On the Android platform, there are multiple navigation designs and they will be described in this section.

\subsection{Navigation drawer}
The hamburger icon at the top left and sliding menu from left to the right is what navigation drawer looks like.
This navigation is suitable for five or more top level screens, or some sort of hierarchical menu \cite{navigationdrawer}.

\subsection{Tabs}
Slidable tabs on top of the screen.
% Recommendation for this type of navigation is having at least two screens.\cite{materialandroid}
Users can click on tab names or just slide left or right in order to navigate between the screens.

\subsection{Bottom navigation}
Bottom navigation consists of icons, usually with text, located at the bottom of the screen.

\section{Conclusion about the main navigation}
The navigation drawer will not be used, because our application does not require five or more main screens nor a hierarchical menu.
Also, with the increasing sizes of mobile phones and most people being right-handed, it is hard to reach the hamburger menu with the right thumb.
There will be lists of items displayed on each of the three main screens.
Those items will be swipeable and having swipeable items on top of swipeable navigation would cause confusion.
Material Design states, that the recommended number of links in the bottom navigation is three to five \cite{bottomnavigation}.
The bottom navigation satisfies our needs and will be used for the main navigation.
The three main screens, containing the bottom navigation, can be seen in \autoref{fig:xdHistory}, \autoref{fig:xdPipelines} and \autoref{fig:xdSettings}.

\begin{figure}\centering
    \begin{minipage}[b]{0.32\textwidth}
    	\includegraphics[width=\textwidth]{pics/xd/Bottom Navigation - executions.png}
    	\caption[History]{History screen design}\label{fig:xdHistory}
    \end{minipage}
    \begin{minipage}[b]{0.32\textwidth}
    	\includegraphics[width=\textwidth]{pics/xd/Bottom Navigation - pipelines.png}
    	\caption[Pipelines]{Pipelines screen design}\label{fig:xdPipelines}
    \end{minipage}
    \begin{minipage}[b]{0.32\textwidth}
    	\includegraphics[width=\textwidth]{pics/xd/Bottom Navigation - settings.png}
    	\caption[Settings]{Settings screen design}\label{fig:xdSettings}
    \end{minipage}
\end{figure}

\section{Lists}
Each of the three main screens will display some sort of list.
For execution screen it is a list of executions, for pipeline screen it is a list of pipelines and for settings screen it is a list of server instances.

All of those lists will have one thing in common and that being the swipe gesture.
When users swipe an item to the left or to the right, the item will be deleted.
This can be seen in \autoref{fig:xdDeletePipeline}.
Users will have the ability to undo this operation for a short period of time.
The undo option can be seen in \autoref{fig:xdUndo}.

Tapping on item from the pipeline screen will open the edit pipeline screen.
Long click on item from execution screen or from pipeline screen will launch the pipeline.

\begin{figure}\centering
    \begin{minipage}[b]{0.32\textwidth}
    	\includegraphics[width=\textwidth]{pics/xd/Bottom Navigation - pipelines – 1.png}
    	\caption[Deleting pipeline]{Deleting pipeline design}\label{fig:xdDeletePipeline}
    \end{minipage}
    \begin{minipage}[b]{0.32\textwidth}
    	\includegraphics[width=\textwidth]{pics/xd/Bottom Navigation - pipelines – 2.png}
    	\caption[Undo option]{Undo option design}\label{fig:xdUndo}
    \end{minipage}
\end{figure}

\section{Edit server instance screen}
While registering new server instance or editing an already registered one, the application needs the address for communication and some name for labeling and better organising.
Users will be able to add a description of the instance, so that there is no pressure to store every information about the instance in the server name.
There could also be an option to ping the server (F-2.6, \autoref{subsec:ping}) to verify the address and a way to cancel the registration/edit.
Because of this, another screen, just for registering/editing server instances, will be added and can be seen in \autoref{fig:xdEditServerInstance}.

\begin{figure}\centering
    \begin{minipage}[b]{0.32\textwidth}
    	\includegraphics[width=\textwidth]{pics/xd/Edit server instance.png}
    	\caption[Edit server instance]{Edit server instance screen design}\label{fig:xdEditServerInstance}
    \end{minipage}
    \begin{minipage}[b]{0.32\textwidth}
    	\includegraphics[width=\textwidth]{pics/xd/Pipeline editor.png}
    	\caption[Edit pipeline screen]{Edit pipeline screen design}\label{fig:xdPipelineEditor}
    \end{minipage}
\end{figure}

\section{Edit pipeline screen}
According to the F-4.3 requirement, described in \autoref{subsec:editpipelinescreen}, there have to be a screen for editing pipelines.
This screen will be displaying pipeline components and drawing links between them.
The preliminary design of this screen can be seen in \autoref{fig:xdPipelineEditor}

% \begin{figure}\centering
%     \begin{minipage}[b]{0.32\textwidth}
%     	\includegraphics[width=\textwidth]{pics/xd/Create new pipeline.png}
%     	\caption[Create new pipeline dialog]{Create new pipeline dialog design}\label{fig:xdCreateNewPipelineDialog}
%     \end{minipage}
%     \begin{minipage}[b]{0.32\textwidth}
%     	\includegraphics[width=\textwidth]{pics/xd/Pipeline editor.png}
%     	\caption[Edit pipeline screen]{Edit pipeline screen design}\label{fig:xdPipelineEditor}
%     \end{minipage}
% \end{figure}

\section{Edit component screen}
This screen has to be created, because each pipeline's component has it's own settings.
In \autoref{fig:xdEditComponent1} and \autoref{fig:xdEditComponent2} is the preliminary design of this screen.

\begin{figure}\centering
    \begin{minipage}[b]{0.32\textwidth}
    	\includegraphics[width=\textwidth]{pics/xd/Edit component - configuration.png}
    	\caption[Edit component screen 1]{Edit component screen 1 design}\label{fig:xdEditComponent1}
    \end{minipage}
    \begin{minipage}[b]{0.32\textwidth}
    	\includegraphics[width=\textwidth]{pics/xd/Edit component - io.png}
    	\caption[Edit component screen 2]{Edit component screen 2 design}\label{fig:xdEditComponent2}
    \end{minipage}
\end{figure}

\section{Notifications}
According to the F-3.1 requirement, described in \autoref{subsec:notifications}, notifications need to be implemented.
The preview of notifications can be seen in \autoref{fig:xdNotifications}.

\begin{figure}\centering
    \begin{minipage}[b]{0.7\textwidth}
    	\includegraphics[width=\textwidth]{pics/xd/Notifications.png}
    	\caption[Notifications]{Notification design}\label{fig:xdNotifications}
    \end{minipage}
\end{figure}
% https://www.youtube.com/watch?v=ugpC98LcNqA
% https://www.youtube.com/watch?v=QrbhPcbZv0I

\section{Software architecture patterns}
Every non trivial software project should follow some architecture design so it always remains maintainable and expandable as much easily as possible.
The project should be modularized, so that any part could be replaced without the need of rewriting the entire project.
And for this reason, there exist architecture patterns.

\subsection{Basic patterns}
% We will go through three of them and explain the conclusion.

% All of these three patterns have one thing in common.
% They structure code into three main layers.
% The names of those layers are present in these pattern's names.

The three basic software architecture patterns will be described here.
The conclusion on which one to pick for our application will be made afterwards.

All of these patterns have one thing in common, that being structuring code into three main layers.
The names of those layers are present in these pattern's names.

\subsubsection{MVC}

\begin{figure}\centering
	\includegraphics[width=1\textwidth]{pics/patterns/bc-mvc2.png}
	\caption[MVC]{Model View Controller}\label{fig:mvc}
\end{figure}

\textbf{Model View Controller}.
View is responsible for displaying data, controller is responsible for getting the user input and model for storing and serving data.
The controller takes user input, it updates the model and then tells the view to update itself. \cite{droidcon}

On the Android platform, the part of application responsible for displaying data is also responsible for dealing with user input.
The pattern can be modified the way that the view gets the user input, sends it to the controller, the controller updated the model and then informs the view to update itself, based on the data from model. \cite{droidcon}

So both view and controller know the model, the view knows controller and the controller knows the view.
That's high consistency and that is a thing to be avoided, especially in the Android world, where forgetting to remove a link can and will cause memory leaks.

And what if the data should be somehow transformed for presentation?
What part should do this presentation transformation, also known as UI logic?
This logic should not be pushed to the view, because it's purpose is only display the given data.
The controller and the model can not posses it neither, because the controller doesn't supply any data to the view and model is responsible for data storing and serving and there is no reason why it should know how to display data.

\subsubsection{MVP}

\begin{figure}\centering
	\includegraphics[width=0.7\textwidth]{pics/patterns/bc-mvp.png}
	\caption[MVP]{Model View Presenter}\label{fig:mvp}
\end{figure}

Following the MVC and the UI logic problem.
What if the view doesn't communicate with the model, but only with controller and the controller was also responsible for taking data from the model and supplying them to the view.
The UI logic could be put in this new controller. This new controller will be called presenter, instead of new controller.
And that's what \textbf{Model View Presenter} is. \cite{droidcon}

But that means the presenter still holds a link to the view and some repetitive code would have to be written because of this. \cite{mvp}
Also, the presenter should not know what parts of the view should be updated after the data changes, because displaying data is not his job.

\subsubsection{MVVM}
\label{subsec:mvvm}

\begin{figure}\centering
	\includegraphics[width=0.7\textwidth]{pics/patterns/bc-mvvm.png}
	\caption[MVVM]{Model View ViewModel}\label{fig:mvvm}
\end{figure}

\textbf{Model View Viewmodel}.
When following the MVP, the presenter knows the view and is executing it's methods when it needs to be updated.
The presenter will no longer know it's view.
Instead it will provide some kind of stream and the view will be able observe the stream, so it can display the data whenever the data changes.
This new presenter will be called viewmodel.

The view now knows the viewmodel, can call it's methods on user input and observe it's data streams.
The viewmodel knows model, observes it's data streams and omits them to the view.
The model doesn't know the view or the viewmodel.

\subsection{Conclusion about picking the pattern}
MVVM suits applications for the Android platform the best.
Google even made some libraries to support MVVM.

\section{Main layers and libraries}
The three main layers of the MVVM will be described here, including a new fourth layer separated from the model.
Some libraries that will be used inside of those layers will also be described here.

\subsection{View}
As stated before, the view is the layer responsible for displaying data.
The way it's implemented is through a combination of standard Kotlin code and XML.
The XML is mostly used to forming the display structure and Kotlin is mostly used to determine what data the view should display, how to update itself and what to do with user input.

In order to work this way, XML and Kotlin have to be somehow linked together.
A library called Data Binding will be used for this purpose.
Compared to other solutions, like Kotlin synthetics, Data Binding offers more features and has a more robust outlook.

\subsection{Viewmodel}
Google has made some architecture components and one of them is exactly for viewmodel.
There is no better way to represent this layer, so this component will be used in our application.

\subsection{Repository}
Repository will represent the part of model responsible for caching and maintaining data stored in the application's local database through communication with the following layer.

\subsection{DAO and network IO}
This will be the fourth layer responsible for storing and loading data and communicating over the internet.

Storing and loading data is a common problem with already existing solutions.
These solutions are called persistence libraries.
One persistence library will be chosen for our application, so the data maintaining doesn't have to be implemented again.
The three popular persistence libraries for the Android platform are Room\cite{room}, Realm\cite{realm} and ObjectBox\cite{objectbox}.
Out of these three, only Room offers the usage of custom written SQL commands, thus providing more flexibility and will be used for our application.

Communicating over the internet is also a common problem, so an existing library will be chosen.
Between commonly used HTTP libraries, not specialised for images, belong Volley\cite{volley} and Retrofit\cite{retrofit}.
Volley is more of a HTTP client and Retrofit makes it very easy to adapt to REST APIs.
Our application will be communicating with the \etl 's REST API and Retrofit is a more fitting solution.

\subsection{LiveData}
Previously in \autoref{subsec:mvvm}, ``some kind of stream'' was mentioned.
For this, an observable data structure is needed and that's what LiveData\cite{livedata} is.
Compared to RxJava\cite{rxjava}, another stream like data structure, LiveData misses some features like working on a background thread and is a bit more complex to use, but LiveData is lifecycle-aware.
Views have something called lifecycle.
Once they're not on screen, all links to them have to removed, otherwise memory leaks occur.
LiveData objects respect these lifecycles, so no additional repetitive code has to be written.
Also, LiveData objects are not really streams, because they do nothing when there is no observer.
If content changes multiple times between  the  screen  refresh  rate,  the  view  gets  only  the  latest  change.
LiveData will be used to propagate data all the way from model, through viewmodel, to view in the MVVM architecture.\todo[inline]{why?}

\section{Conclusion}

\begin{figure}\centering
	\includegraphics[width=1\textwidth]{pics/bc-architecture.png}
	\caption[Architecture]{Diagram consisting of MVVM and server instances}\label{fig:architecture}
\end{figure}

Repository will communicate with the DAO and Network layer in order to download, store and load data.
Viewmodel will communicate with repository and offer data ready to display to the view layer.
View layer will display data and react to user input by forwarding it to the viewmodel.
Passing data to the view layer will be handed using LiveData.
The summary can be seen in \autoref{fig:architecture}\todo[inline]{use autoref}

%\chapter{Realization}
\chapter{Implementation}
\label{chap:implementation}
In this chapter we will cover the most important things that are needed to implement the application.
% In this chapter we will cover the most important libraries and algorithms that were used in the implementation process.

\section{Libraries}
While coding the application, there's a need to learn about some libraries. \todo[inline]{why?}

\subsection{Room}
Room makes it easy to work with application's internal database.
For each entity you want to store in database, you define an entity class, which is just a normal kotlin class with some annotations.
For the simplest entity class, you just need to annotate the whole class with \verb|@Entity| and annotate primary key either with \verb|@PrimaryKey| or with multiple primary keys, you can list them as an argument in \verb|@Entity| annotation.
In order to get objects to and from database, you need DAO classes.
The magic is that you declare abstract methods with annotations, optionally with some SQL syntax and Room implements them for you.
Those methods return either the entities, lists of entities or LiveData instances of entities or lists.

\subsection{LiveData}
LiveData is an observable data structure.
Components of view layer can observe those variables, which means that they can react to content changes.
Under the hood, it's a bit more complex.
If content changes multiple times between the screen refresh rate, the view gets only the latest change.
Before observing, the view passes it's own instance of lifecycle, so the LiveData instance knows, when the view is not active or destroyed, so the LiveData instance can delete the reference to this view and stop updating this observer, because of performance and memory leaks.
One of the best things about this library is, that Room can also return LiveData instances, so each time something from the selected group changes, the view knows about it.

\subsection{Retrofit}
Retrofit is a nice android library for making network requests.
The setup is similar to Room DAO classes.
For each API call, you define an abstract method.
It also works very well with OkHttp library, which is different http request library, that among everything supports the basic authentication.

\subsection{Coroutines}
Coroutines are kotlin's way to improve multithreaded programming.
It introduces a new keyword \verb|suspend| that can be written in front of methods.
Methods with the \verb|suspend| keyword can't be called the normal way, but can be called from other \verb|suspend| methods or launched via higher order methods of \verb|CoroutineScope|.

\subsection{WorkManager}
WorkManager library is used to handle services.
Service is a part of an application that can run in background and can be started even when the application is not running.
We will be using this for checking execution statuses.

\subsection{Gson}
Gson makes serializing and deserializing java/kotlin objects really easy.
There are no requirements for the code to change in order to use Gson.
It will be used while transforming incoming network content to objects we can work with.

\subsection{Varvet's QR code scanner}
Google made an API for working with QR codes and Varvet made it even easier.
Varvet made an open source project BarcodeReaderSample and published it under MIT license.

\subsection{DraggableView by hyuwah}
Pipeline's components should be freely movable across the canvas while editing the pipeline.
That can be achieved by making some regular view (button or image) movable.
This is often achieved by a lot of unnecessary code and that's why we will be using this library that should make it easy.

\section{Interesting logic}
This section will describe some of the more complex internal logic.

\subsection{Undo operations}
There are some undo options in planning.
We want the user to be able to undo pipeline deletion or execution deletion.
Just scheduling the sending of the request doesn't do the trick.
We will be creating and storing marks for every scheduled deletion.
So when the application is killed for some reason before the delete request is sent, we can take marks stored in database and send the delete requests at the another start of the application.
Marks can also be used for filtering pipelines and executions that will be showing to the user.
If user undo the deletion, the mark is deleted and the scheduling of the delete request is cancelled.

%\begin{figure}\centering
%    \begin{minipage}[b]{0.45\textwidth}
%    	\includegraphics[width=\textwidth]{pics/undo/delete_action.png}
%    	\caption[Delete action]{Delete action}\label{fig:undoDelete}
%    \end{minipage}
%    \begin{minipage}[b]{0.45\textwidth}
%    	\includegraphics[width=\textwidth]{pics/undo/undo_action.png}
%    	\caption[Undo action]{Undo action}\label{fig:undoUndo}
%    \end{minipage}
%\end{figure}
%\begin{figure}\centering
%    \begin{minipage}[b]{0.45\textwidth}
%    	\includegraphics[width=\textwidth]{pics/undo/db_clean.png}
%    	\caption[Finish interrupted delete actions]{Finish interrupted delete actions}\label{fig:undoDbClean}
%    \end{minipage}
%\end{figure}
\begin{figure}\centering
    \begin{minipage}[b]{0.45\textwidth}
    	\includegraphics[width=\textwidth]{pics/undo/delete_action.png}
    	\caption[Delete action]{Delete action}\label{fig:undoDelete}
    \end{minipage}
    \begin{minipage}[b]{0.45\textwidth}
    	\includegraphics[width=\textwidth]{pics/undo/undo_action.png}
    	\caption[Undo action]{Undo action}\label{fig:undoUndo}
    \end{minipage}
    \begin{minipage}[b]{0.45\textwidth}
    	\includegraphics[width=\textwidth]{pics/undo/db_clean.png}
    	\caption[Finish interrupted delete actions]{Finish interrupted delete actions}\label{fig:undoDbClean}
    \end{minipage}
\end{figure}

\subsection{Execution status}
In order to know, when an execution is finished or cancelled, the application has to periodically check for it.
By looking at the duration length of executions from the public demo server, it is obvious that most of them don't surpass ten seconds.
But looking at executions from one private server, they least for hours and some of them even days.
We will be using this checking strategy:
\begin{itemize}
    \item For the first 10 seconds, check every second.
    \item Next 50 seconds, check every 5\textsuperscript{th} second.
    \item Then check once every hour.
\end{itemize}

\section{Tests}
It is a good practise to write special code that can check whenever parts of the application work as intended.
This code is simply called tests.
This is not only useful to test the application parts when they are written, but these tests could be run in future to ensure that possible code changes didn't broke the functionality of previously written parts.

We will be using libraries Hamcrest and MockK.
Hamcrest will be used for matching lists and MockK for mocking.

\subsection{Local unit tests}
These are tests that require only JVM.
They don't need any part of the android framework, so there's no need to run them on an android device, which makes them fast.
Parsing objects from JSON and back can and will be tested here.

\begin{figure}\centering
	\includegraphics[width=1\textwidth]{pics/coverage.png}
	\caption[Test Coverage]{Picture with local test coverage}\label{fig:coverage}
\end{figure}
We will test the model layer here, but since we don't have access to the database, repository and DAOs will be excluded from these tests.
The coverage, as displayed here: \ref{fig:coverage}, is 89 \% for classes and 85 \% for methods.
70 \% to 80 \% is commonly considered good coverage.

\subsection{Instrumented unit tests}
Instrumented tests require some part of the android framework, so they run in background on an android device.
Previously mentioned library Room requires a part of android framework, so it can be used here.
Repositories will be tested here too.

Unfortunately, the coverage could not be put into operation here.
After researching how coverage can be done on instrumented tests, most of the tutorials are bound to the JaCoCo library, which caused some errors during commissioning attempts.

\section{Documentation}
Looking at a code after some time may feel strange and not very intuitive and that's why documentations exist.
\todo[inline]{nepište to jako příběh. "In this section, we provide the user and developer documentation.}

\subsection{UI Documentation}
No matter how good the UI is, some users may still struggle with it.
For that case, there will be a UI documentation at the front page of the application's github page, alongside with couple of video tutorials going through previously written use cases.

\subsection{Developer documentation}
In kotlin, there are a documenting type of comment called KDoc.
There is also a plugin called Dokka, that can construct a website based on the KDoc comments, documenting application's code.

It's also a good idea to write a piece of documentation by hand, explaining the internal operations from a greater distance, so other developers can have a more bearable understanding of the application when they follow up on development.

Links to both of those will be accessible on the project's github page.

\chapter{Deployment}
Google Play will be used for the deployment of this application.
It is an android application store which is already installed on most of android phones.
Using Google Play is far more better than just downloading and installing the app from some storage.
Some of the advantages are simplicity of updating and the option to leave reviews.
The application will be available to everyone using android version 5.0 (Lollipop) or greater.

\begin{conclusion}
	%sem napište závěr Vaší práce
	% Our goals were a list of requirements, a look on existing solutions, design the application, the application itself and documentation.
% 
% \begin{itemize}
%     \item In the requirements engineering chapter we got together use cases, scenarios, and made a structured document containing requirements.
%     \item In chapter existing solutions we came to the conclusion, that there is currently no existing solution to our problem, so we need to create the application.
%     \item We designed the UI and chose the architecture pattern for our application.
%     \item Application has been implemented and tested.
%     \item We wrote documentation for our ETL client.
% \end{itemize}

The goals were a list of requirements, a look on existing solutions, design the application, the application itself, documentation and testing.

\begin{itemize}
    \item Use cases, scenarios and requirements were put together in the analysis chapter.
    \item The existence of no current solution for our requirements was verified in the existing solutions chapter.
    \item The UI was designed in the UI design chapter and the software architecture was designed in the architecture design chapter.
    \item The application was created during the implementation chapter.
    \item The documentations were created and described in the documentation chapter.
    \item The application tests were described in the test chapter.
\end{itemize}

In the future, the application may be improved with a better UI.
Some components are currently not supported and in the future, it could be solved by a cooperation of the application developer(s) and the server developers.
\end{conclusion}

%\bibliographystyle{csn690}
%\bibliography{mybibliographyfile}
\begin{thebibliography}{9}

\bibitem[1]{material} Material Design [online], Google LLC, [cit. 2020-12-06], Available from: \url{https://material.io}

\bibitem[2]{droidcon} MUNTENESCU, Florina and Touchlab, Droidcon NYC 2016 - A Journey Through MV Wonderland (updated), Youtube [video], Youtube LLC, 2016, [cit. 2020-12-06], Available from: \url{https://youtu.be/QrbhPcbZv0I}

\bibitem[3]{mvp} LEIJDEKKERS, Peter and ANZ Coders, Android MVP vs MVVM and the winner is..., Youtube [video], Youtube LLC, 2018, [cit. 2021-03-26], Available from: \url{https://youtu.be/ugpC98LcNqA}

\bibitem[4]{room} Save data in a local database using Room | Android Developers [online], Google LLC, [cit. 2021-04-10], Available from: \url{https://developer.android.com/training/data-storage/room}

\bibitem[5]{livedata} LiveData Overview | Android Developers [online], Google LLC, [cit. 2021-04-10], Available from: \url{https://developer.android.com/topic/libraries/architecture/livedata}

\bibitem[6]{retrofit} Retrofit [online], Square, Inc., [cit. 2021-04-10], Available from: \url{https://square.github.io/retrofit/}

\bibitem[7]{varvet} Android QR Code Reader Made Easy — Varvet [online], Varvet, [cit. 2021-04-10], Available from: \url{https://www.varvet.com/blog/android-qr-code-reader-made-easy/}

\bibitem[8]{draggable} DraggableView [online], Muhammad Wahyudin, [cit. 2021-04-10], Available from: \url{https://github.com/hyuwah/DraggableView}

\end{thebibliography}

\appendix

\chapter{List of abbreviations used}
% \printglossaries
\begin{description}
	\item[API] Application Programming Interface
\item[DAO] Data access object
\item[GUI] Graphical user interface
\item[JSON] JavaScript Object Notation
\item[JVM] Java virtual machine
\item[REST] Representational State Transfer
\item[UI] User interface
\item[URI] Uniform Resource Identifier
\item[XML] Extensible markup language
\end{description}

\chapter{Contents of the enclosed CD}

\begin{figure}
	\dirtree{%
		.1 readme.txt\DTcomment{brief description of the contents of the CD}.
	    .1 doc/.
	    .2 index.html\DTcomment{developer documentation}.
	    .2 videos.html\DTcomment{video tutorials online}.
	    .1 videos/\DTcomment{video tutorials}.
	    .1 LinkedPipesAndroidClient/\DTcomment{application's source codes}.
	    .2 README.md\DTcomment{UI documentation and OS requirements}.
		.1 ETLClient.apk\DTcomment{application package}.
		.1 thesis/\DTcomment{\LaTeX{} source codes of this thesis}.
		.1 thesis.pdf\DTcomment{this thesis as a PDF}.
	}
\end{figure}

%upravte podle skutecnosti
%
%\begin{figure}
%	\dirtree{%
%		.1 readme.txt\DTcomment{stručný popis obsahu CD}.
%		.1 exe\DTcomment{adresář se spustitelnou formou implementace}.
%		.1 src.
%		.2 impl\DTcomment{zdrojové kódy implementace}.
%		.2 thesis\DTcomment{zdrojová forma práce ve formátu \LaTeX{}}.
%		.1 text\DTcomment{text práce}.
%		.2 thesis.pdf\DTcomment{text práce ve formátu PDF}.
%		.2 thesis.ps\DTcomment{text práce ve formátu PS}.
%	}
%\end{figure}

\end{document}
